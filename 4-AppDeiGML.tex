\chapter{Applicazione dei Grafi Multilivello ai Sogni} \label{cap:applicazione-dei-grafi-multilivello-ai-sogni}

Come testimoniato dalle loro svariate applicazioni, nella loro generalità e semplicità, i grafi si sono da sempre
dimostrati strutture in grado di catturare l'essenza di tanti aspetti della realtà dalla più variegata natura.
Che siano reti sociali, reti di trasporto, reti di comunicazione, reti biologiche, reti semantiche o reti di calcolo,
essi permettono di fornire un utile modello per la rappresentazione di dati complessi.
Il fatto che i grafi multi-livello siano proprio basati su grafi conferisce loro una notevole versatilità che permette
di applicarli ad una vasta gamma di problemi, tra cui l'analisi di testi in linguaggio naturale,
ed in particolare, a racconti di sogni.
Tuttavia il potenziale dei grafi multi-livello, che si basa sulla loro capacità di catturare aspetti della struttura
di un grafo ad un livello di astrazione superiore, è ciò che li rende particolarmente interessanti e che può
fornire un valore aggiunto all'analisi delle relazioni di elementi discreti, come le parole di un testo.

Si noti, infatti, che nel contesto generale dell'analisi di grafi, la contrazione realizzata attraverso
l'uso di grafi multi-livello potrebbe essere utile per:
\begin{itemize}
    \item[a.1] Ridurre la complessit\`a dell'analisi strutturale di un grafo, sia che esso debba essere processato
    attraverso algoritmi costosi, sia che esso debba essere graficamente visualizzato, rendendolo pi\`u facilmente
    interpretabile ed evidenziandone le caratteristiche strutturali di interesse.
    \item[a.2] Studiare l'interrelazione di caratteristiche strutturali di un grafo, che rappresenti la navigabilit\`a
    di uno spazio basato su componenti fortemente connesse, cicli, cricche ecc.
    Si noti che gli spazi generati da livelli superiori al primo non potrebbero essere altrimenti ottenuti se
    non attraverso un approccio multi-livello.
    \item[a.3] Stabilire il grado di connettivit\`a di un grafo, individuando la rilevanza (intesa come il numero di
    insiemi componente che rappresentano i supernodi), il numero e la dimensione delle sue contrazioni.
    \item[a.4] Misurare il grado di complessit\`a dello spazio rappresentativo del grafo, in base al numero
    di nodi e archi presenti nelle contrazioni: grafi derivanti da specifici domini tendono ad avere un certo
    grado di complessit\`a legato ad un concetto spaziale, come successivamente mostrato nell'
    esempio~\ref{fig:les-miserables-graph}.
    \item[a.5] Valutare l'influenza di singoli nodi e archi appartenenti al grafo di base sui livelli superiori
    del grafo multi-livello, eseguendo analisi della sensitivit\`a e della robustezza.
    Sebbene non siano presentate in questa tesi, è possibile definire delle procedure che permettano di aggiornare
    coerentemente la struttura memorizzata di un grafo multi-livello all'aggiunta e rimozione di singoli nodi e
    archi al livello base.
\end{itemize}

Nel contesto dell'analisi di testi, in particolare di racconti di sogni, con l'eventuale ausilio di strumenti di
elaborazione del linguaggio naturale, la contrazione realizzata attraverso l'uso di grafi multi-livello potrebbe
essere utile per:
\begin{itemize}
    \item[b.1] Individuare contesti sintattici e possibilmente semantici di parole e frasi, evidenziando l'interrelazione
    e la distanza tra gruppi di parole e frasi.
    \item[b.2] Valutare la somiglianza di singoli racconti, individuando le macro-caratteristiche strutturali comuni e
    la loro differenziazione tra più sognatori diversi.
    \item[b.3] Individuare pattern ricorrenti di parole e insiemi di parole su un corpus pi\`u ampio di testi, con
    eventuale ausilio di strumenti statistici.
    \item[b.4] Stabilire la natura della connettività del grafo delle parole in base al numero di insiemi componente
    (gruppi di parole) e di nodi presenti nelle contrazioni.
    \item[b.5] Permettere un confronto automatico dei pattern strutturali con grafi multi-livello che rappresentino
    un controllo, con l'eventuale ausilio di algoritmi che valutano il grado di somiglianza di grafi.
\end{itemize}

In questo capitolo verranno, quindi, esplorate le possibili modalità di applicazione dei grafi multi-livello allo scopo
di analisi di racconti di sogni, portando dei casi di studio reali in relazione ai punti b.1 e b.2, con l'obiettivo di
evidenziarne le capacità nell'ambito di un'analisi sintattica e semantica.



Per rendere chiaro questo aspetto, si prenda come esempio il grafo multi-livello $M = (G, \langle f_{C_1}, f_{C_2}\rangle)$
rappresentato in figura~\ref{fig:les-miserables-graph}, il cui grafo $G$ rappresenta il grafo delle relazioni di
co-apparizione dei personaggi del romanzo \textit{Les Misérables} di Victor Hugo,
e le cui funzioni di contrazione $f_{C_1}$ e $f_{C_2}$ rappresentano rispettivamente le funzioni di contrazione
per cricche non reciproche e componenti fortemente connesse.
Appare evidente di come la struttura del grafo $G$ sia stata contratta con un elevato tasso di contrazione,
producendo una struttura notevolmente più semplice e facilmente interpretabile. Per via degli schemi di
contrazione scelti, appare evidente come la struttura originale del grafo abbia rivelato la sua ordinatezza ai livelli
superiori: in media i personaggi del romanzo possono apparire assieme ad una cerchia ristretta di altri personaggi,
ad eccezione di personaggi principali che risultano collegarsi a questi gruppi più o meno isolati di nodi.
Questo rispecchia in parte la natura dello spazio bidimensionale in cui i personaggi sono collocati: personaggi
appartenenti a luoghi distanti difficilmente appariranno insieme. I personaggi principali, in quanto seguiti
nella narrazione nel mentre che si spostano nello spazio, permettono di rompere questa dimensionalità, e risultano
essere collegati a nodi ``distanti'' tra loro.

