\chapter{Aggiornamento Dinamico}\label{cap:aggiornamento-dinamico}

Avendo presentato nel Capitolo~\ref{cap:procedure-contrazione} quanto necessario per affrontare il problema
della costruzione di un grafo multilivello, si vogliono ora considerare le implicazioni che deriverebbero
dalla modifica della struttura del grafo alla base di una gerarchia già stabilita.
In questa capitolo verrà affrontato il problema dell'aggiornamento dinamico di un grafo multilivello, inteso
come la definizione delle operazioni necessarie affinché la sequenza di grafi decontraibili già costruiti
$G_0$, $G_1$ \ldots $G_k$ espressa da un grafo multilivello $M = (G, \Gamma)$ possa essere aggiornata in modo coerente
rispetto all'aggiunta e rimozione di archi e nodi al grafo $G$.
Si vuole precisare che questo problema non ricopre particolare rilevanza dal punto di vista algebrico o dichiarativo,
in quanto sostituire un grafo $G$ con un grafo simile $H$ non comporterebbe altro che alla sostituzione
del grafo multilivello $M$ con un nuovo grafo multilivello $N = (H, \Gamma)$.
Si potrebbe invece dire che il problema risulta interessante dal punto di vista algoritmico: si consideri una possibile
rappresentazione in memoria di un grafo multilivello in cui la struttura di ciascun grafo decontraibile
$G_0$, $G_1$ \ldots $G_k$ sia effettivamente memorizzata.
In tal caso l'aggiornamento dinamico si pone come obiettivo quello di individuare delle procedure che risultino
essere computazionalmente vantaggiose rispetto alla ricostruzione dell'intera gerarchia di grafi decontraibili da zero
mediante l'applicazione delle procedure di contrazione nella loro completezza. \newline
Più precisamente, si vogliono definire degli algoritmi da eseguire a ciascun livello $i$ della gerarchia che,
data una modifica al grafo decontraibile $G_{i-1}$, permettano di far propagare tale modifica al grafo
decontraibile $G_i$ affinché rimanga una sua contrazione in accordo alla funzione di contrazione $f_{C_i}$.

A riguardo, valgono le seguenti importanti considerazioni legate agli schemi di contrazione considerati precedentemente:
\begin{itemize}
    \item L'operazione di aggiunta di un nodo a $G_{i-1}$ comporta semplicemente l'aggiunta di un nodo a $G_i$ per
    qualunque schema di contrazione adottato al livello $i$ tra quelli fon'ora considerati.
    Si noti infatti che l'aggiunta di un nodo non comporta la creazione di nuovi archi tra i nodi del livello $i$, e
    gli schemi di contrazione per componenti fortemente connesse, cricche e circuiti sono unicamente sensibili alla
    presenza di archi tra nodi.
    Questo non \`e certamente vero in generale, come potrebbe essere il caso di schemi di contrazione per insiemi
    indipendenti dove, dato un grafo $G = (V, E)$, un suo insieme indipendente di nodi (o \textit{anticricca}) è un
    sottoinsieme $S \subseteq V$ tale per cui nessun nodo in $S$ è adiacente ad un altro nodo in $S$.
    \item Le operazioni di aggiunta e rimozione di un arco al livello $i-1$ possono comportare delle modifiche
    sostanziali al livello $i$ per gli schemi considerati, in quanto definiti proprio sulla base delle relazioni.
    Inoltre, tali modifiche sono determinate dallo specifico schema di contrazione adottato, e saranno determinate
    da algoritmi specifici per ciascuno schema.
    \item L'operazione di rimozione di un nodo da $G_{i-1}$ pu\`o essere realizzata attraverso la rimozione
    progressiva degli archi entranti ed uscenti dal nodo da rimuovere.
    Successivamente si pu\`o procedere con la rimozione del nodo stesso, che, nel caso degli schemi considerati,
    comporta semplicemente la rimozione del relativo supernodo in $G_i$.
    \item Operazioni di unione con un altro grafo possono essere realizzate attraverso la combinazione delle
    operazioni precedenti.
\end{itemize}

Per questi motivi, le attenzioni di questo capitolo saranno rivolte principalmente alle operazioni di aggiunta e
rimozione di un arco per ciascuno dei tre schemi di contrazione considerati, ovvero per componenti fortemente connesse,
cricche massimali e circuiti semplici. \newline

Ciascuno degli algoritmi sar\`a definito come una procedura locale al livello del grafo $G_i = (V, E)$ della gerarchia
che prenda in input un arco $(a, b) \in \mathcal{E}$ (che sia stato rimosso o aggiunto al livello inferiore, che modifichi
il dizionario $T$ del livello $i$ in accordo all'operazione descritta e fornisca in output al livello superiore della
gerarchia insiemi di nodi $V^+, V^- \subseteq V$ e di archi $E^+,E^- \subseteq E$ che rappresentino le modifiche
apportate al livello $i$, rispettivamente come insiemi di nodi e archi da aggiungere e rimuovere. \newline
A seguito della modifica del dizionario $T$, la modifica effettiva della struttura del grafo $G_i$ sar\`a realizzata
attraverso la procedura UPDATE-GRAPH, che si occuper\`a di aggiungere e rimuovere nodi e archi in accordo alle modifiche
apportate al dizionario $T$. \newline
In caso di pi\`u modifiche al livello $i-1$, e quindi pi\`u chiamate consecutive di ciascuno degli algoritmi di
aggiornamento, potrebbe essere necessario utilizzare la procedura UPDATE-GRAPH pi\`u volte per mantenere coerente la
struttura del grafo $G_i$ con le modifiche apportate al dizionario $T$, qualora gli algoritmi specifici di aggiornamento
richiedano di conoscere la struttura aggiornata del grafo $G_i$ per poter operare correttamente.

%TODO: Forse qui o dopo?
Definiremo poi in che modo le modifiche potranno essere propagate efficientemente tra i livelli della gerarchia.


\subsection{Aggiornamento dinamico per componenti fortemente connesse}
\label{subsec:aggiornamento-dinamico-per-componenti-fortemente-connesse}

\nlparagraph{Aggiunta di un arco}
Tra gli schemi di contrazione considerati, quello per componenti fortemente connesse \`e l'unico che permette
di far coincidere ciascun nodo con uno ed un solo insieme $C$ nel dizionario $T$.
Questo significa che l'aggiunta di un arco $(a, b)$ al livello $i-1$ pu\`o comportare una modifica sostanziale alla
struttura del grafo $G_i$ al livello $i$ solo se le componenti fortemente connesse a cui appartengono $a$ e $b$ e,
quindi, i supernodi associati, non coincidono e non esiste gi\`a un superarco tra di esse. \newline
In tal caso, l'aggiunta dell'arco $(a, b)$ pu\`o comportare la fusione delle due componenti fortemente connesse
presenti nel grafo $G_{i-1}$ e corrisponenti ai nodi $a$ e $b$ in una nuova componente fortemente connessa.

L'algoritmo qui presentato permette di realizzare la logica descritta, realizzando l'aggiornamento degli insiemi
componente tracciati al livello $i$ a seguito dell'aggiunta di un arco $(a, b)$ al livello $i-1$.

\begin{algorithm}[H]
    \caption{UPDATE-ADDED-EDGE-SCC($G_i$, $(a, b)$)}\label{alg:add-edge-scc}
    \begin{algorithmic}[1]
        \State $u \coloneqq a.supernode$
        \State $v \coloneqq b.supernode$
        \If{$u == v$}
            \State ADD-EDGE-IN-SUPERNODE($u$, $(a, b)$)
        \Else
            \State ADD-EDGE-IN-SUPEREDGE($(u, v)$, $(a, b)$)
            \If{$|dec_E((u, v))| == 1$}
                \State $Q \coloneqq$ REACH-VISIT($G_i$, $v$, $u$)
                    \If{$Q \neq \emptyset$}
                        \For{$w \in Q$}
                            \State REMOVE-COMPONENT-SET($T$, $T[\text{any node in } V_w]$)
                        \EndFor
                        \State $C_{new} \coloneqq \bigcup_{w \in Q} V_w$
                        \State ADD-COMPONENT-SET($T$, $C_{new}$)
                    \EndIf
            \EndIf
        \EndIf
    \end{algorithmic}
\end{algorithm}

Nel caso in cui le componenti fortemente connesse di $a$ e $b$ siano distinte, l'algoritmo procede ad eseguire la
procedura REACH-VISIT, che consiste di una visita in profondit\`a sul grafo contratto $G_i$ a partire dal supernodo
di $b$, restituendo l'insieme di tutti i supernodi che possono raggiungere il supernodo di $a$.

\begin{algorithm}[H]
    \caption{REACH-VISIT($G$, $s$, $t$)}\label{alg:reach-visit}
    \begin{algorithmic}[1]
        \For {$v \in V$}
            \State $Reach[v] \coloneqq NIL$
        \EndFor
        \State $Reach[t] \coloneqq TRUE$
        \State REACH-DFS($G$, $s$)
        \State $Q \coloneqq \{v \in G.V \mid reach[v] == TRUE\}$
        \State \Return $Q$
    \end{algorithmic}
\end{algorithm}
\begin{algorithm}[H]
    \caption{REACH-DFS($G$, $u$)}\label{alg:reach-dfs}
    \begin{algorithmic}[1]
        \If{$Reach[u] == NIL$}
            \State $Reach[u] \coloneqq FALSE$
            \For {$v \in G.Adj[u]$}
                \State REACH-DFS($G$, $v$)
                \State $Reach[u] \coloneqq Reach[u] \lor Reach[v]$
            \EndFor
        \EndIf
    \end{algorithmic}
\end{algorithm}

La procedura REACH-VISIT etichetta tutti i nodi in $G_i$ con una flag $Reach$ che indica se il nodo pu\`o raggiungere
il supernodo di $a$ ed \`e al contempo raggiungibile dal supernodo di $b$.
Quando $Q$ non \`e vuoto, l'algoritmo procede a rimuovere gli insiemi corrispondenti ai supernodi trovati, che saranno almeno
due se $a \neq b$, e ad aggiungere un nuovo insieme $C_{new}$ che rappresenta l'unione delle componenti connesse
individuate. \newline

Si noti che l'algoritmo, facendo affidamento alla struttura di $G_i$, necessita che tale struttura sia
aggiornata rispetto alle modifiche apportate al dizionario $T$.
Pertanto l'algoritmo UPDATE-GRAPH dovr\`a essere eseguito prima di una nuova esecuzione di UPDATE-ADDED-EDGE-SCC.

\subparagraph{Proposizione 9.1.1}
Sia $G_{i-1} = (\mathcal{V}, \mathcal{E})$ un grafo diretto, sia $G_i$ la contrazione di $G_{i-1}$ secondo
lo schema di contrazione per componenti fortemente connesse.
Sia $(a, b) \in (\mathcal{V} \times \mathcal{V})$ l'arco passato alla procedura UPDATE-ADDED-EDGE-SCC tale che
$(u \neq v)$ e $(u, v) \notin E$ dove $u = a.supernode$ e $v = b.supernode$, sia
$H_{i-1} = (G_{i-1}.V, G_{i-1}.E \cup {(a, b)})$ il grafo risultante dall'aggiunta dell'arco $(a, b)$ a $G_{i-1}$.
\newline
Al termine dell'esecuzione dell'algoritmo UPDATE-ADDED-EDGE-SCC su $(a, b)$ e $G_i$, l'insieme $Q$ contiene tutti e
soli i supernodi tali per cui $\bigcup_{w \in Q} V_w$ \`e una nuova componente fortemente connessa
in $H_{i-1}$, o \`e vuoto se tale componente non esiste.

\subparagraph{Dimostrazione}
Per le propriet\`a delle condensazioni, il grafo $G_i$ ha componenti fortemente connesse con un solo supernodo,
e quindi \`e un grafo connesso aciclico.
Si ha pertanto che per qualsiasi $w, x \in V$, non pu\`o essere che $w \rightsquigarrow x$ e $x \rightsquigarrow w$
in $G_i$, in quanto $w$, $x$ e tutti i supernodi che siano raggiungibili da $w$ e che possono raggiungere
$x$ sarebbero in una unica componente fortemente connessa.
Lo stesso vale per $u$ e $v$.

Si noti inoltre che, considerando $x \rightsquigarrow x \quad \forall x \in V$ , in generale si ha
\begin{equation*}
    \forall w, x \in V \quad w \rightsquigarrow x \quad  \Longleftrightarrow \quad
    \exists y \in V \quad \text{t.c.} \quad w \rightsquigarrow y \land y \rightsquigarrow x
\end{equation*}

Si consideri ora l'insieme $Q \subseteq V$.
Esso pu\`o contenere solo supernodi $w$ tali per cui esiste un cammino $w \rightsquigarrow v$, in quanto \`e
costruito a partire dai nodi che hanno nodo adiacente $v$ o un qualsiasi altro nodo che pu\`o raggiungerlo.
Inoltre essendo costruito da una visita a partire da $u$, $Q$ contiene tutti e soli i supernodi $w$ tali per cui
esistono almeno due cammini $u \rightsquigarrow w$ e $w \rightsquigarrow v$. \newline

Si ottiene quindi
\begin{equation*}
    Q \neq \emptyset \quad \Longleftrightarrow \quad
    u \rightsquigarrow v \text{ in } G_i \quad \Longleftrightarrow \quad
    \lnot (v \rightsquigarrow u) \text{ in } G_i
\end{equation*}

Pertanto, supponendo $\lnot (u \rightsquigarrow v)$ in $G_i$, all'aggiunta dell'arco $(u, v)$ certamente si ha
$u \rightsquigarrow v$ con $\{u, v\}$ gli unici supernodi $w \in V$ tali per cui
$u \rightsquigarrow w \land w \rightsquigarrow v$. \newline
Supponendo invece $v \rightsquigarrow u$ in $G_i$, si deve avere  $\lnot u \rightsquigarrow v$ in $G_i$.
All'aggiunta dell'arco $(u, v)$ tali condizioni rimangono invariate, e quindi $Q = \emptyset$. \newline
Pertanto si conclude che $Q$ contiene tutti e soli i supernodi tali per cui $Q$ \`e una nuova componente fortemente
connessa composta da pi\`u supernodi in $H_{i} = (G_i.V, G_i.E \cup \{(u, v))\}$, o \`e vuoto se tale componente non
esiste.

\subparagraph{Complessità}
La complessit\`a dell'algoritmo UPDATE-ADDED-EDGE-SCC \`e $O(|V|+|E|) + O(\mathcal{V})$, dove $V$ e $E$ sono rispettivamente
sono gli insieme dei nodi e degli archi del grafo $G_i$ e $\mathcal{V}$ \`e l'insieme dei nodi di $G_{i-1}$.\newline
La complessit\`a di $O(|V|+|E|)$ \`e dovuta alla visita in profondit\`a eseguita dalla procedura REACH-VISIT, che esegue
la procedura REACH-DFS al pi\`u una volta per ciascun nodo del grafo $G_i$, in quanto ogni nodo $w$ viene visitato
solo quando $Reach[w]$ non \`e $NIL$, e ad ogni visita di un nodo $w$ si imposta $Reach[w]$ ad un valore booleano
permanentemente, mentre ogni arco uscente viene valutato al pi\`u una volta, in quanto arco uscente di un solo nodo.
Il ciclo for a riga 10 pu\`o modificare al pi\`u una volta ogni riga di $T$, e quindi una volta per ogni nodo in
$\mathcal{V}$, in quanto ogni nodo appartiene ad una sola componente connessa, e quindi ad un solo insieme $C$.
Pertanto la complessit\`a del ciclo \`e $O(\mathcal{V})$.
Stessa cosa pu\`o essere detta per il calcolo dell'unione a riga 13 e per l'esecuzione di ADD-SUBSET a riga 14.

\paragraph{Rimozione di un arco}
La rimozione di un arco $(a, b)$ al livello $i-1$ pu\`o comportare la divisione di una componente fortemente connessa
solo nel caso in cui l'arco $(a, b)$ appartenga all'insieme $dec_E(v)$ per qualche supernodo $v$ in $G_i$.
Altra condizione necessaria affinch\`e la componente venga divisa \`e che $a$ non possa pi\`u raggiungere $b$ a
seguito della rimozione dell'arco $(a, b)$. \newline

A seguito della verifica positiva della prima delle due condizioni, l'algoritmo provvede alla verifica della seconda
controllando con una visita in profondit\`a in $G_v$ a partire da $a$ che $b$ non sia nell'insieme $C \subseteq dec_V(v)$
dei nodi raggiungibili.
In tal caso, l'algoritmo procede ad individuare le nuove componenti connesse interne alla componente connessa del
supernodo $v$.
Come dimostrato in seguito, \`e possibile assumere che i nodi in $C$ siano una delle nuove componenti fortemente
connesse al livello $i-1$.

L'algoritmo per la rimozione di un arco $(a, b)$ al livello $i-1$ pu\`o, quindi, essere definito come segue:

\begin{algorithm}[H]
    \caption{UPDATE-REMOVED-EDGE-SCC($G_i$, $(a, b)$)}\label{alg:remove-edge-scc}
    \begin{algorithmic}[1]
        \State $u \coloneqq a.supernode$
        \State $v \coloneqq b.supernode$
        \If{$u \neq v$}
            \State REMOVE-EDGE-IN-SUPEREDGE($(u, v)$, $(a, b)$)
        \Else
            \State REMOVE-EDGE-IN-SUPERNODE($u$, $(a, b)$)
            \State Make a DFS visit in $G_u$ from $a$
            \State $C \coloneqq $ the set of nodes reachable from $a$
            \If{$V_u \neq C$}
                \State Let $H$ be the subgraph of $G_u$ induced by $V_u \setminus C$
                \State $H' \coloneqq$ COMPONENT-CONTRACT($H$)
                \State REMOVE-SUBSET($T$, $V_u$)
                \State ADD-SUBSET($T$, $C$)
                \For{$S \in \{V_w \mid w \in H'.V\}$}
                    \State ADD-SUBSET($T$, $S$)
                \EndFor
            \EndIf
        \EndIf
    \end{algorithmic}
\end{algorithm}

Anche in questo caso , l'algoritmo pu\`o modificare sensibilmente la struttura di $G_i$ nel caso in cui la rimozione
dell'arco comporti la disgregazione di una componente fortemente connessa, e pertanto la procedura UPDATE-GRAPH
dovrebbe essere eseguita in seguito a tale evenienza in virt\`u delle successive chiamate agli algoritmi di
aggiornamento.

\subparagraph{Proposizione 9.1.2}
Sia $G_{i-1} = (\mathcal{V}, \mathcal{E})$ un grafo diretto, sia $G_i$ la contrazione di $G_{i-1}$ secondo
lo schema di contrazione per componenti fortemente connesse.
Sia $(a, b) \in (\mathcal{V} \times \mathcal{V})$ l'arco passato alla procedura UPDATE-REMOVED-EDGE-SCC tale che
$u = a.supernode = b.supernode$, sia $H_{i-1}$ il grafo risultante dalla rimozione dell'arco $(a, b)$ da $G_{i-1}$.
Sia $C$ l'insieme di nodi raggiungibili da $a$ in $H_{i-1}[V_u]$.

\begin{enumerate}[(i)]
    \item Se $b \in C$, allora $dec_V(u)$ \`e una componente fortemente connessa anche in $H_{i-1}$.
    \item Se $b \notin C$, allora $C$ \`e una nuova componente fortemente connessa in $H_{i-1}$.
\end{enumerate}

\subparagraph{Dimostrazione}
Si nota innanzitutto che, per definzione di $G_u$ e $H_{i-1}[V_u]$, si ha che la rimozione dell'arco $(a, b)$
mantiene inalterata la raggiungibilit\`a di $a$ e di tutti i nodi in $dec_V(u)$ rispetto a $b$.
\begin{equation*}
        \forall c \in V_u
        \quad c \rightsquigarrow a \land b \rightsquigarrow c
        \text{ in } H_{i-1}[V_u]
\end{equation*}

Per dimostrare (i), assumiamo per ipotesi $b \in C$.
Questo significa che $a \rightsquigarrow b$ in $H_{i-1}[V_u]$.
Per via di (1), se $b \in C$, tutti i nodi in $H_{i-1}[V_u]$, che possono raggiungere $a$, possono raggiungere
anche $b$ attraverso $a$.
Inoltre, potendo raggiungere $b$, tutti nodi possono raggiungere tutti gli altri nodi attraverso $b$.
Pertanto $V_u$ \`e una componente fortemente connessa. \newline
Per dimostrare (ii), assumiamo per ipotesi $b \notin C$.
Questo significa che $a$ non pu\`o raggiungere $b$ in $H_{i-1}[V_u]$ e che quindi $V_u$ non pu\`o essere
una componente fortemente connessa.
Inoltre, poich\`e $C$ \`e l'insieme dei nodi raggiungibili da $a$ in $H_{i-1}[V_u]$, tutti
i nodi in $C$ sono mutualmente raggiungibili, e per via di (1) possono raggiungere $a$ in quanto $C \subseteq V_u$.
Pertanto si conclude che $C$ \`e una nuova componente fortemente connessa.

\subparagraph{Complessità}
La complessit\`a dell'algoritmo UPDATE-REMOVED-EDGE-SCC \`e $O(|\mathcal{V}_u|+|\mathcal{E}_u|)$, dove $\mathcal{V}_u$ e
$\mathcal{E}_u$ sono rispettivamente l'insieme dei nodi e degli archi del grafo $G_u$, con $u$ l'eventuale
supernodo in $G_i$ in comune ad entrambi i nodi agli estremi dell'arco $(a, b)$ da aggiungere.
Si noti infatti che nel caso la condizione a riga 3 sia verificata, l'algoritmo rimane di una complessit\`a
$O(1)$.
Nel caso, invece, $a.supernode = b.supernode = u$,la complessit\`a \`e data principalmente dalla visita in profondit\`a
in $G_u$ a riga 7, di costo $O(|\mathcal{V}_u|+|\mathcal{E}_u|)$ e dalla procedura COMPONENT-CONTRACT a riga 12, che
come discusso in precedenza ha complessit\`a $O(|\mathcal{V}_u|+|\mathcal{E}_u|)$, in quanto $H$ \`e un sottografo di
$G_u$.
Inoltre il ciclo for a riga 15 mantiene una complessit\`a di $O(|\mathcal{V}_u|)$, in quanto scorre una
partizione di nodi in $H$ ed esegue al pi\`u una modifica alla tabella $T$ per ogni nodo in $H$ nella totalit\`a
delle chiamate di ADD-SUBSET.