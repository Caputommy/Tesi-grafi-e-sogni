\section{Analisi delle differenze sintattiche tra sognatori} \label{sec:analisi-delle-differenze-sintattiche-tra-sognatori}

Come anticipato nel Capitolo~\ref{cap:mondo-dei-sogni}, nel contesto psichiatrico la rappresentazione di testi sui
sogni attraverso grafi si \`e rivelato un'utile strumento a supporto della diagnosi di disturbi come la schizofrenia
o il disturbo bipolare~\cite{mota2014graph}, nonch\'e la possibilit\'a di predire l'insorgenza di patologie in
anticipo rispetto alla diagnosi clinica~\cite{mota2017thought}.

Tali studi evidenziando di come l'attenzione ad aspetti sintattici della descrizione di sogni possano fornire
informazioni significative sullo stato mentale di un individuo rispetto a quando le stesse tecniche sono applicate
ad altri tipi di testi prodotti dallo stesso.

Aspetti di rilievo del grafo delle parole, ovvero il grafo ottenuto dal report orale di un sogno in cui
ogni parola \`e rappresentata come un nodo e ogni connessione temporale tra parole consecutive come un arco diretto,
sono stati la frequenza di utilizzo delle parole, la presenza di cicli su parole ricorrenti, la connettivit\'a delle
parole e la loro organizzazione in componenti e componenti fortemente connesse. \newline

\subsection{Pre-elaborazione con strumenti di NLP}\label{subec:pre-elaborazione-con-NLP}
Per poter procedere con l'analisi dei sogni di Emma affinché si potesse costruire un grafo delle parole
da cui poter cogliere aspetti semantici, è stato necessario effettuare una pre-elaborazione dei testi
attraverso gli strumenti tipici dell'elaborazione del linguaggio naturale.
L'NLP (in inglese \textit{Natural Language Processing}) è di fatti quella disciplina a metà tra intelligenza artificiale
e linguistica che si occupa di individuare i metodi di elaborazione e analisi di dati che si presentano sotto forma di
linguaggio naturale, ovvero di linguaggi usati dell'essere umano. \newline

La pre-elaborazione ha quindi previsto l'applicazione della seguente sequenza di fasi al corpus di 1218 sogni
originario:
\begin{enumerate}
    \item Pulizia del testo. \newline \noindent
          Questa fase ha incluso la rimozione degli spazi bianchi superflui, la standardizzazione della
          formattazione del testo e la correzione di errori ortografici o di incoerenze. È stata inoltre eseguita la
          rimozione della punteggiatura, snellendo ulteriormente i dati testuali. Tali procedure di pulizia sono
          essenziali per garantire la qualità e la coerenza dei dati in ingresso, migliorando così l'affidabilità
          dell'analisi.
    \item Tokenizzazione. \newline \noindent
          Questa fase ha comportato la suddivisione del testo in parole singole dette \textit{token},
          passaggio fondamentale nell'elaborazione del linguaggio naturale, creando la base per le fasi successive.
    \item Rimozione delle stop words. \newline \noindent
          Le \textit{stop words} sono parole comuni (come ``the'', ``is'', ``at'', ``which'' e ``on'') che tipicamente
          hanno un ruolo significativo nei compiti di NLP, specialmente in quelli legati ad analisi semantiche.
          In questo caso, la loro rimozione può certamente aiutare a concentrare l'analisi sulle parole più ricche di
          contenuto nelle narrazioni dei sogni, riducendo significativamente il rumore nei dati e mettendo in evidenza
          i termini più salienti.
    \item Lemmatizzazione. \newline \noindent
          La lemmatizzazione è un processo che riduce le parole alla loro forma base, chiamata \textit{lemma}.
          Ad esempio, ``running'' verrebbe lemmatizzato in ``run'' e ``better'' in ``good''.
          Questo passaggio aiuta a ridurre le diverse declinazioni delle parole ai loro concetti di base, riducendo
          la dimensionalità dei dati testuali e rivelando, potenzialmente, schemi sottostanti in modo più chiaro.
    \item Costruzione del grafo delle parole. \newline \noindent
          Infine, è stato costruito un grafo diretto basato sulle singole parole all'interno delle sequenze
          di lemmi associate ad ogni sogno. Tali parole, in quanto prese singolarmente rispetto alla sequenza
          di appartenenza, si definiscono \textit{unigrammi}. In questo grafo, quindi, ogni nodo rappresenta
          un unigramma univoco, con una proprietà ``peso'' che indica il numero di occorrenze di quell'unigramma
          nell'intero corpus di sogni.
          Gli archi orientati del grafo collegano gli unigrammi dei lemmi che appaiono seguitamente nella
          sequenza associata ad ogni sogno. Per questo motivo, ad ogni arco corrisponde un \textit{bigramma}, ovvero
          un accostamento ordinato di due parole.
          Il peso di ciascun arco rappresenta il numero di occorrenze di quello specifico accostamento ordinato nel testo.
          In questo modo, la struttura del grafo cattura non solo le relazioni sequenziali tra le parole nei sogni,
          ma anche la frequenza e la forza di queste parole e relazioni.
\end{enumerate}