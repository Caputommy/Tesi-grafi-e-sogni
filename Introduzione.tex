\chapter*{Introduzione}\addcontentsline{toc}{chapter}{Introduzione}

Sin dall'era della psicoanalisi Freudiana, l'analisi dei sogni ha rappresentato un territorio misterioso
ed affascinante che si proponeva di rivelare aspetti nascosti della psiche umana.
Nonostante la scarsa considerazione scientifica nel corso degli anni, oggi l'analisi del contenuto dei sogni
quantifica aspetti specifici per condurre analisi statistiche.
Studi hanno mostrato che i sogni sono correlati allo stato psicologico, e che disturbi mentali come depressione e
schizofrenia influenzano significativamente i contenuti onirici.
Recenti analisi dei report di sogni legate alla teoria dei grafi suggeriscono che questi possano rivelare differenze
psicologiche altrimenti non osservabili.
Inoltre, grazie a progetti innovativi come quello di \textit{Sognario}, l'analisi dei report di sogni si affaccia
all'epoca dei Big Data, permettendo di analizzare un numero sempre crescente di sogni.
Si propone quindi, in questa tesi, un sistema di analisi basato su \textit{grafi multi-livello},
una struttura dati basata su grafi che permette di rappresentare e analizzare i sogni attraverso una serie di livelli
di astrazione.
Dopo una breve introduzione al mondo della ricerca sui sogni e ai fondamenti della teoria dei grafi,
si definiranno i concetti teorici alla base della struttura dati proposta e si illustreranno gli algoritmi
fondamentali derivanti dalla sua progettazione.
Infine, attraverso dei casi di studio, si dimostrerà come questa struttura dati possa essere utilizzata per
estrapolare informazioni legate alla sintassi e alla semantica delle parole a partire da
grandi moli di dati testuali sui sogni.