\subsection{Condensazione in un grafo decontraibile}\label{subsec:condensazione-grafo-decontraibile}

La procedura segue la logica dell'algoritmo di Kosaraju per l'individuazione delle componenti strettamente connesse
descritto in [1], effettuando due visite in profondit\`a, una sul grafo originale $G$ e una sul grafo trasposto
$G^T$, in cui la seconda \`e effettuata secondo l'ordinamento della prima.
Durante la seconda visita in profondit\`a viene costruito il grafo $G\mathcal{'}$ ottenuto come contrazione del
grafo decontraibile $G$.

    Il seguente algoritmo descrive la seconda visita in profondit\`a, che include la costruzione del grafo
    contratto di $G$. \newline
    I parametri passati alla procedura ricorsiva DFS-VISIT' includono
    \begin{itemize}
        \item il grafo decontraibile originale  $G$
        \item il nodo da  visitare $u$
        \item il supernodo $\alpha_i$, che rappresenta una componente fortemente connessa di $G$ che include $u$
        \item i due insiemi da costruire $V_{\alpha_i}, E_{\alpha_i}$, che insieme definiscono il grafo $\alpha_i.dec = (V_{\alpha_i}, E_{\alpha_i})$ 
		associato al supernodo $\alpha_i$.
	    \item l'insieme di superarchi $\mathfrak{E}$, che vengono definiti nel corso della visita.
		Gli archi in $\epsilon.dec$ con $\epsilon \in \mathfrak{E}$ vengono aggiunti quando si incontrano nodi appartenenti
        ad altre componenti connesse gi\`a visitate.
    \end{itemize}

    \begin{algorithm}[!H]
    \caption{DFS$\mathcal{'}$($G$)}\label{alg:cap3}
    \begin{algorithmic}[1]
        \For {each $v \in G.V$}
            \State $v.color =$ WHITE
        \EndFor
        \State $\mathfrak{V} \coloneqq \emptyset$
        \State $\mathfrak{E} \coloneqq \emptyset$
        \State i $\coloneqq$ 0
        \For {each $u \in G.V$ in decreasing order by $u.f$}
            \If {$u.color ==$ WHITE}
                \State let $\alpha$ be a new super-node
                \State $\mathfrak{V} \coloneqq \mathfrak{V} \cup \{\alpha_i\}$
                \State $V_{\alpha_i} \coloneqq \emptyset$
                \State $E_{\alpha_i} \coloneqq \emptyset$
                \State DFS-VISIT'($G, u, \alpha, V_{\alpha_i}, E_{\alpha_i}, \mathfrak{E}$)
                \State $\alpha.dec \coloneqq$ $(V_{\alpha_i}, E_{\alpha_i})$
                \State $i \coloneqq i + 1$
            \EndIf
        \EndFor
        \State $G\mathcal{'} \coloneqq (\mathfrak{V}, \mathfrak{E})$
        \State \textbf{return} $G\mathcal{'}$
    \end{algorithmic}
\end{algorithm}
    \begin{algorithm}[!H]
    \caption{DFS-VISIT'($G, u, \alpha, V_{\alpha}, E_{\alpha}, \mathfrak{E}$)}\label{alg:cap}
    \begin{algorithmic}[1]
        \State $u.color \coloneqq$ GREY
        \State $u.supernode \coloneqq \alpha$
        \State $V_{\alpha} \coloneqq V_{\alpha} \cup \{u\}$
        \For {each $v \in G.adj[u]$}
            \If {($v.color ==$ WHITE $\vee$ $v.supernode == \alpha$)}
                \State $E_{\alpha} \coloneqq E_{\alpha} \cup  (v ,u)$
            \EndIf
            \If{($v.color ==$ WHITE)}
                \State DFS-VISIT'($G, v, \alpha, V_{\alpha}, E_{\alpha}, \mathfrak{E}$)
            \ElsIf{($v.color ==$ BLACK $\wedge$ $v.supernode \ne \alpha$)}
            \If{$(v.supernode, \alpha) \notin \mathfrak{E}$}
                \State $(v.supernode, \alpha).dec \coloneqq \emptyset$
                \State $\mathfrak{E} \coloneqq \mathfrak{E} \cup \{(v.supernode, \alpha)\}$
            \EndIf
            \State $(v.supernode, \alpha).dec \coloneqq (v.supernode, \alpha).dec \cup \{(v, u)\}$
            \EndIf
        \EndFor
        \State $u.color \coloneqq$ BLACK
    \end{algorithmic}
\end{algorithm}

    \newpage

    \newpage
    In particolare:
    \begin{itemize}
        \item i nodi $v$ che soddisfano la condizione a riga 5 di DFS-VISIT$\mathcal{'}$ fanno parte della stessa componente di $u$.
        Per questi nodi l'arco originale del grafo non trasposto $(v, u)$ viene aggiunto al grafo corrispondente al
        supernodo $\alpha$ a riga 6;
        \item i nodi $v$ che soddisfano la condizione a riga 10 di DFS-VISIT$\mathcal{'}$ sono nodi che fanno parte di altre
        componenti fortemente connesse gi\`a visitate che sono adiacenti a $u$.
        Per questi nodi, si considerano i superarchi ottenuti dai rispettivi supernodi di $v$ e $u$, aggingendoli a
	    $\mathfrak{E}$ se necessario, (riga 13), e aggiungendo l'arco originale $(v, u)$ all'insieme di archi
        rappresentato dal superarco (riga 15).
    \end{itemize}

    \paragraph{Complessit\`a dell'algoritmo}
    L'algoritmo COMPONENT-CONTRACT applicato ad un grafo $G = (V, E)$  mantiene una complessit\`a temporale di
    $\Theta(V + E)$, in quanto:
    \begin{itemize}
        \item A riga 1 viene eseguita una ricerca in profondit\`a tradizionale, quindi con costo $\Theta(V + E)$.
        \item A riga 2 viene costruito il grafo trasposto $G^T = (V, E^T)$, dove $E^T = \{(u, v) \mid (v, u) \in E\}$ \`e
        dato dagli archi in $E$ con orientamento invertito.
        Anche in questo caso, la sua costruzione pu\`o avvenire in un tempo $O(V + E)$, iterando prima sull'insieme dei
        nodi di $G$ e poi sui suoi archi, invertendone l'orientamento.
        L'ordinamento dei nodi di $G^T$ pu\`o essere realizzato nel corso della visita a riga 1 attraverso la costruizione
        di una lista concatenata, per cui i nodi possono essere aggiunti in testa alla lista in tempo $O(1)$  man mano
        che viene loro assegnata l'etichetta di fine visita $u.f$.
        \item A riga 3 viene eseguita la procedura DFS' che corrisponde ad una visita in profondit\`a di $G^T$ abbinata
        alle operazioni di costruzione del grafo decontraibile.
        In particolare, le operazioni che vengono effettuate in aggiunta a quelle di una visita in profondit\`a consistono
        nei semplici assegnamenti alle righe 4\textendash6, 10\textendash12, 14\textendash15 e 18 di DFS' e alle righe
        2\textendash3 di DFS-VISIT'.
        Inoltre sono previsti dei controlli e dei relativi assegnamenti alle righe 5\textendash7 e 10\textendash16 che,
        ancora una volta, aggiungono solamente costi costanti per ogni arco in $G^T$.
        Il costo computazionale della procedura rimane $\Theta(V + E)$.
    \end{itemize}

    \subsection{Correttezza dell'algoritmo}\label{subsec:correttezza-dell'algoritmo}

    \paragraph{Lemma 3.2.1}
    Sia $G=(V,E)$ un grafo decontraibile input dell'algoritmo COMPONENT-CONTRACT, sia $G\mathcal{'}$ il grafo decontraibile
    in output, sia $u$ un nodo in $V$, sia $v$ un nodo adiacente a $u$ in $(G)^T$, sia $\alpha$ il supernodo di $u$.
    Le condizioni sul nodo $v$ alle righe 5 e 10 di DFS-VISIT' sono tra loro mutualmente esclusive e ricoprono tutti i
    possibili casi.
    In particolare:
    \begin{enumerate}[(i)]
        \item ($v.color ==$ WHITE $\vee$ $v.supernode == \alpha$) $\implies$ il nodo $v$ ha lo stesso supernodo di $u$ al
        termine dell'algoritmo.
        \item ($v.color ==$ BLACK $\wedge$ $v.supernode \ne \alpha$) $\implies$ il nodo $v$ non ha lo stesso supernodo di
        $u$ al termine dell'algoritmo.
    \end{enumerate}

    \paragraph{Dimostrazione}
    Si noti, innanzitutto, che l'attributo $supernode$ una volta assegnato non viene pi\`u modificato fino al termine
    dell'algoritmo, in quanto esso pu\`o essere assegnato solamente a riga 2 di DFS-VISIT' e DFS-VISIT' viene richiamato
    una ed una sola volta per ciascun nodo. \newline

    Per dimostrare (i) ci limitiamo a dimostrare $v.color ==$ WHITE $ \implies v.supernode == u.supernode$ e
    $v.supernode == \alpha \implies v.supernode == u.supernode$.
    \begin{enumerate}
        \item Assumiamo $v.color ==$ WHITE. Questo significa che esso non ha ancora alcun supernodo assegnato e che gli 
		verr\`a assegnato il supernodo $\alpha$ nella chiamata di DFS-VISIT' a riga 9, in quanto la condizione a riga 8
        \`e verificata $(H_1)$.
        \item Assumiamo $v.supernode == \alpha$.
        Banalmente, essendo $u.supernode = \alpha$ si ottiene $v.supernode == u.supernode$ $(H_2)$.
    \end{enumerate}
    Per $(H_1)$ e $(H_2)$ abbiamo dimostrato il punto (i).

    Per dimostrare (ii) assumiamo $v.color ==$ BLACK $(H_3)$ e $v.supernode \ne \alpha$ $(H_4)$.
    Dato $(H_3)$, il nodo $v$ \`e gi\`a stato visitato e deve, pertanto, avere un supernodo assegnato.
    Per $(H_4)$, questo supernodo non pu\`o essere $\alpha = u.supernode$, pertanto si conclude
    $v.supernode \neq u.supernode$. \newline

    Le conclusioni dei punti (i) e (ii) risultano essere, quindi, mutualmente esclusive e ricoprono tutti i possibili
    casi sullo stato del nodo $v$ ed $u$ al termine dell'algoritmo.

    \paragraph{Correttezza dell'algoritmo}
    Al termine dell'esecuzione dell'algoritmo su un grafo decontraibile $G\coloneqq(V, E)$ avremo costruito un grafo
    decontraibile $G\mathcal{'} \coloneqq (\mathfrak{V}, \mathfrak{E})$ dove:
    \begin{enumerate}[(i)]
        \item $G\mathcal{'}$ \`e una contrazione di $G$
        \item Per ogni supernodo $\alpha \in \mathfrak{V}$, con $\alpha.dec = (V_{\alpha}, E_{\alpha})$, $V_{\alpha}$ \`e una componente fortemente
        connessa di $G$
    \end{enumerate}

    \paragraph{Dimostrazione}
    La propriet\`a (i) pu\`o essere dimostrata notando che:
    \begin{itemize}
        \item La prima propriet\`a delle contrazioni \`e rispettata, in quanto ogni nodo $u \in V$ viene visitato una ed una
        sola volta da DFS-VISIT' e viene aggiunto ad un certo insieme $V_{\alpha}$ per qualche $\alpha \in \mathfrak{V}$.
        Inoltre, per ogni supernodo $\alpha \in \mathfrak{V}$, $V_{\alpha} \neq \emptyset$ in quanto esso dovr\`a contenere
        almeno il nodo $u$ della prima chiamata di DFS-VISIT' a riga 13 di DFS'.

        \item La seconda propriet\`a delle contrazioni \`e rispettata.
        Per il Lemma 3.2.1 le condizioni sul nodo $v$ alle righe 5 e 10 di DFS-VISIT' sono tra loro mutualmente
        esclusive e ricoprono tutti i possibili casi al termine dell'algoritmo: o $v$ ha lo stesso supernodo di $u$ o
        non ha lo stesso supernodo di $u$.
        Quindi si pu\`o notare che tutti gli archi in $E$ sono considerati ognuno una ed una sola volta dal ciclo for a
        riga 4 di DFS-VISIT'. Per ognuno di questi archi $(v, u)$, o vengono aggiunti ad un certo insieme $E_{\alpha}$
        per qualche $\alpha$ quando $u$ e $v$ sono tali per cui $u.supernode = v.supernode = \alpha$, o vengono aggiunti ad
        un insieme $e.dec$ per qualche $e \in \mathfrak{E}$ quando $u$ e $v$ sono tali per cui
        $u.supernode \neq v.supernode$.
        Questo vuol dire che
        $\{ E_\alpha \mid \alpha \in \mathfrak{V}$, $\alpha.dec = (V_\alpha, E_\alpha)\} \cup \{ \epsilon .dec \mid \epsilon \in \mathfrak{E}\}$
        \`e un ricoprimento di $E$ formato di insiemi a due a due disgiunti. \newline
        Inoltre, la propriet\`a per cui $\emptyset \notin \{\epsilon .dec \mid \epsilon \in \mathfrak{E}\}$ \`e garantita dal fatto che
        l'esecuzione della riga 13 di DFS-VISIT' \`e sempre succeduta dall'esecuzione della riga 16, per cui non pu\`o
        esistere $\epsilon \in \mathfrak{E}$ tale per cui $\epsilon .dec = \emptyset$.
    \end{itemize}

    La propriet\`a (ii) pu\`o essere dimostrata notando che l'agoritmo utilizza la stessa logica dell'algoritmo per
    l'individuazione delle componenti connesse descritto in [1].
    In particolare, \`e possibile dimostrare per induzione che alla k-esima chiamata di DFS-VISIT', le componenti
    connesse $C_i$ con $ i = 0,\ldots, k-1$, tali per cui $f(C_i)$ \`e il tempo di fine visita massimo tra i nodi della
    componente i-esima e, quindi, tali per cui $f(C_i) > f(C_k)$ $\forall i$, sono state gi\`a visitate e assegnate a
    degli insiemi $V_{\alpha_i}$ per dei supernodi $\alpha_i$.
    Inoltre, per il Lemma 22.15 descritto in [1], le componenti $C'$ tali per cui $f(C') < f(C_k)$ non possono essere
    visitate durante la visita dell'albero di $C_k$, in quanto in $G^T$ non possono esistere archi della forma $(v ,u)$
    con $v \in C' $ e $ u \in C_k$.
    Per l'assegnamento a riga 3 di DFS-VISIT', ogni nodo visitato viene aggiunto all'insieme $V_{\alpha_i}$, il quale cambia
    a riga 11 di DFS' per ogni nuovo albero di ricerca in profondit\`a che viene percorso.
    Pertanto al termine dell'algoritmo $V_{\alpha_i}$ rappresenter\`a una componente fortemente connessa di $G$ per ogni $\alpha_i$.