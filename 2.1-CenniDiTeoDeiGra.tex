\section{Cenni di Teoria dei Grafi}\label{sec:cenni-di-teoria-dei-grafi}

Un grafo \`e una struttura matematica costruita su un insieme di elementi in cui coppie di elementi possono essere
in relazione tra loro.
I grafi possono essere orientati o non orentati, a seconda che esista una direzione o un ordine tra le coppie
di elementi che si trovano in relazione.
In questa sezione si concentrerà l'attenzione sui grafi orientati, in quanto alla base della definizione dei
grafi multi-livello, oltre che essere più generali: i grafi non orientati possono sempre essere rappresentati
come particolari grafi orientati.

\subsection{Grafi orientati}\label{subsec:grafi-orientati}

\begin{definition}[Grafo orientato]
    Un \textbf{grafo orientato} $G$ \`e una coppia $(V, E)$, dove:
    \begin{itemize}
        \item $V  = \{v_1, v_2, \ldots, v_n\}$ \`e un insieme finito non vuoto di elementi detti \textbf{nodi}
        (o \textbf{vertici}).
        \item $E = \{(v_i, v_j) \mid v_i, v_j \in V\} \subseteq V \times V$ \`e un insieme di coppie ordinate di
        nodi dette \textbf{archi} (o \textbf{spigoli}).
    \end{itemize}
\end{definition}

Nelle rappresentazioni grafiche dei grafi orientati, i nodi sono solitamente rappresentati
come cerchi o punti, mentre gli archi come frecce.
Nella figura~\ref{fig:directed-graph-example} \`e mostrato un esempio di grafo orientato con insieme di nodi
$V = \{v_1, v_2, v_3, v_4, v_5, v_6, v_7\}$ e insieme di archi $E = \{(v_1, v_2), (v_2, v_1), (v_2, v_3), (v_3, v_1)
(v_3, v_4), (v_4, v_4), (v_5, v_6), (v_6, v_5)\}$. \newline
Si noti che sono ammessi \textbf{cappi}, ovvero archi che collegano un nodo a se stesso, ma nella normale nozione
di grafo orientato non sono ammessi archi multipli tra due nodi.

\begin{figure}[t]
    \centering
    \begin{tikzpicture}
    % Define the nodes
    \node[mynode] (1) at (0, 0) {$v_1$};
    \node[mynode] (2) at (2, 2) {$v_2$};
    \node[mynode] (3) at (4, 0) {$v_3$};
    \node[mynode] (4) at (2, -2) {$v_4$};
    \node[mynode] (5) at (6, 2) {$v_5$};
    \node[mynode] (6) at (8, 0) {$v_6$};
    \node[mynode] (7) at (6, -2) {$v_7$};

    % Draw the edges
    \draw[myarrow] (1) -- (2);
    \draw[myarrow] (2) to[out=180, in=90] (1);
    \draw[myarrow] (4) to[out=225, in=135, looseness=5] (4); % This edge is a self-loop, corrected below
    \draw[myarrow] (2) -- (3);
    \draw[myarrow] (3) -- (4);
    \draw[myarrow] (3) -- (1);
    \draw[myarrow] (5) to[out=0, in=90] (6);
    \draw[myarrow] (6) -- (5);
\end{tikzpicture}
    \vspace{-10pt}
    \caption{Esempio di grafo orientato}
    \label{fig:directed-graph-example}
\end{figure}

Le cardintalit\`a degli insiemi di nodi e archi di un grafo orientato sono $|V|$ e $|E|$,
e vengono dette rispettivamente \textbf{ordine} e \textbf{dimensione} del grafo.

Essendo definiti su un insieme di elementi e di archi, possono essere definite relazioni di inclusione tra grafi.
Un grafo $G' = (V', E')$ \`e un \textbf{sottografo} di $G = (V, E)$, e lo si indica con $G' \subseteq G$, se $V' \subseteq V$
e $E' \subseteq E$.

Inoltre, dato un certo insieme $V' \subseteq V$, si definisce il sottografo di $G$ \textbf{indotto}
da $V'$, e lo si indica con la notazione $G[V']$, il grafo avente come insieme di nodi $V'$ e come insieme di archi
l'insieme di tutti gli archi in $G$ che rappresentino relazioni tra tali nodi, ovvero il grafo $G' = (V', E')$ dove
$E' = \{ (u, v) \in E : u, v \in V'\}$.

Essendo il contenuto informativo rilevante di un grafo orientato contenuto nei suoi archi, e quindi nelle relazioni
tra nodi, il concetto di eguaglianza tra grafi orientati non risulta banale.
Una relazione tra grafi orientati, utile per valutare la loro equivalenza in termini di informazione espressa,
\`e l'isomorfismo (dal greco, \textit{iso} = uguale e \textit{morph\`e} = forma).
Così come per tutte le strutture matematiche, intuitivamente, due grafi si dicono \textbf{isomorfi} quando per ogni
parte della struttura di uno esiste una corrispondente parte della struttura dell'altro, e viceversa.
Formalmente, due grafi orientati $G = (V, E)$ e $H = (W, F)$ si dicono isomorfi, e lo si indica con $G \cong H$, se
esiste una biiezione $f: V \rightarrow W$ tale per cui $(u, v) \in E$ se e solo se $(f(u), f(v)) \in F$ per ogni
$u, v \in V$.

\nlparagraph{Archi e nodi}\label{par:archi-e-nodi-di-un-grafo-orientato}

A seguire alcune definizioni relative ai nodi e agli archi di un grafo orientato: \newline

Sia $(u, v) \in E$ un arco di un grafo orientato $G = (V, E)$, allora:
\begin{itemize}
    \item l'arco $(u, v)$ \textbf{esce} dal nodo $u$ ed {entra} nel nodo $v$.
        Ad esempio, gli archi uscenti dal nodo $v_2$ nel grafo della figura~\ref{fig:directed-graph-example}
        sono $(v_2, v_1)$ e $(v_2, v_3)$, mentre l'unico arco entrante nel nodo $v_5$ \`e $(v_6, v_5)$.
    \item l'arco $(u, v)$ si dice \textbf{incidente} in entrambi i vertici $u$ e $v$.
    \item il nodo $v$ \`e detto \textbf{adiacente} al nodo $u$, in quanto esiste un arco $(u, v) \in E$.
\end{itemize}

Sia $v \in V$ un nodo di un grafo orientato $G = (V, E)$, allora:
\begin{itemize}
    \item il \textbf{grado uscente} di un nodo $v$ \`e il numero di archi che escono da $v$.
    \item il \textbf{grado entrante} di un nodo $v$ \`e il numero di archi che entrano in $v$.
    \item il \textbf{grado} di un nodo $v$ \`e la somma del grado uscente e del grado entrante di $v$.
\end{itemize}

Le informazioni sui gradi dei nodi di un grafo possono essere utilizzati per descriverne le loro caratteristiche
strutturali.
Ad esempio, un grafo si può chiamare una \textbf{stella} quando, informalmente, ogni nodo del grafo ha un
grado pari a $1$, ad eccezione di un nodo, il centro della stella, con grado maggiore o uguale a $1$.

\subsection{Cammini e Connessione tra nodi}\label{subsec:cammini}

\nlparagraph{Cammini}

I cammini sono concetti fondamentali della teoria dei grafi e sono alla base di molti algoritmi e problemi noti. \newline

Sia $G = (V, E)$ un grafo orientato, siano $u, v \in V$ due nodi di $G$, allora un \textbf{cammino} da $u$ a $v$ in $G$
\`e una sequenza ordinata di nodi $\langle v_0, v_1, \ldots, v_k \rangle$ tale che $(v_i, v_{i+1}) \in E$ per ogni
$i = 0, 1, \ldots, k-1$ con $v_1 = u$ e $v_k = v$.
La \textbf{lunghezza} $k$ di un cammino \`e data dal numero di archi che lo compongono.
Ad esempio, $\langle v_1, v_2, v_3, v_4 \rangle$ \`e un cammino di lunghezza 3 nel grafo della
figura~\ref{fig:directed-graph-example}.

Nel caso di grafi pesati, quindi aventi archi con un valore associato che ne indichi un costo, si definisce
\textbf{costo} di un cammino la somma dei pesi degli archi che lo compongono.
Inoltre, un \textbf{cammino minimo} tra una coppia di nodi $u$ e $v$ è un cammino di costo (o lunghezza, nel caso
di grafi non pesati) minore o uguale a quello di ogni altro cammino tra gli stessi nodi.

Se esiste un qualsiasi cammino $p$ da $u$ a $v$ in $G$, allora si dice che il nodo $v$ \`e \textbf{raggiungibile} da
$u$ attraverso $p$ in $G$, e questo pu\`o essere indicato con la notazione $u \overset{p}{\rightsquigarrow} v$. \newline

A seguire alcune definizioni relative ai cammini su un grafo orientato:

\begin{itemize}
    \item Un cammino si dice \textbf{semplice} se non contiene nodi ripetuti, ad eventuale eccezione del primo e
    dell'ultimo nodo.
    \item Un cammino si dice \textbf{elementare} se non contiene archi ripetuti. Si noti che in un grafo orientato, un
    cammino semplice \`e sempre elementare.
    \item Un cammino $\langle v_0, v_1, \ldots, v_k \rangle$ di lunghezza $k \geq 1$ si dice \textbf{ciclo} se $v_1 = v_k$, ovvero se il
    suo nodo iniziale coincide con il suo nodo finale.
    Un \textbf{ciclo semplice} \`e un cammino in cui tutti i nodi sono distinti ad eccezione del primo e dell'ultimo
    nodo, mentre un \textbf{ciclo elementare} (o \textbf{circuito}) \`e un ciclo in cui tutti gli archi sono distinti.
    Ad esempio, nel grafo in figura~\ref{fig:directed-graph-example}, il cammino $\langle v_1, v_2, v_3, v_1 \rangle$
    \`e un circuito semplice di lunghezza 3.
    Inoltre, un grafo diretto che non contiene cicli semplici \`e detto grafo diretto \textbf{aciciclico}.
    \item Un cammino si dice \textbf{cammino hemiltoniano} nel grafo $G$ se attraversa ogni nodo di $G$ esattamente
    una volta.
    \item Un cammino $\langle v_0, v_1, \ldots, v_k \rangle$ si dice \textbf{ciclo hemiltoniano} nel grafo $G$ se
    esso \`e un ciclo e ogni nodo di $G$ appare una ed una sola volta tra i nodi $\langle v_0, v_1, \ldots,
    v_{k-1}\rangle$ e $v_k = v_0$.
    \item Un cammino si dice \textbf{cammino euleriano} nel grafo $G$ se attraversa ogni arco di $G$ esattamente una
    volta.
    \item Un cammino si dice \textbf{ciclo euleriano} nel grafo $G$ se esso \`e un ciclo e ogni arco di $G$ appare una
    ed una sola volta tra gli archi del ciclo.
\end{itemize}

\nlparagraph{Connessione tra nodi}

Una caratteristica importante dei grafi orientati, basata sul concetto di raggiungibilit\`a e adiacienza, e quindi
sui cammini, \`e la connessione dei suoi nodi.
\newline

Un grafo orientato $G = (V, E)$ si dice \textbf{fortemente connesso} se per ogni coppia di nodi $u, v \in V$ esiste un
cammino da $u$ a $v$.
In un tale grafo, quindi, ogni nodo \`e mutualmente raggiungibile da ogni altro nodo. \newline
Le \textbf{componenti fortemente connesse} di un grafo sono le classi di equivalenza dei nodi secondo la relazione
``essere mutualmente raggiungibili''.
Ad esempio, nel grafo in figura~\ref{fig:directed-graph-example}, le componenti fortemente connesse sono
$\{\{v_1, v_2, v_3\}, \{v_4\}, \{v_5, v_6\}, \{v_7\}\}$.
Si noti che l'insieme di nodi $V$ di un grafo fortemente connesso \`e per definizione una unica componente connessa.
\newline

Un maggiore grado di connessione tra nodi \`e dato dalla presenza di singoli archi tra ogni coppia di nodi anzich\`e
di cammini. \newline
Un grafo orientato $G = (V, E)$ si dice \textbf{completo} se esiste un arco $(u, v) \in E$ per ogni coppia di nodi
distinti $u, v \in V$.
In un tale grafo, quindi, ogni coppia di nodi distinti \`e adiaciente. \newline
Le \textbf{cricche} di un grafo sono le classi di equivalenza dei nodi secondo la relazione
``essere mutualmente adiacienti''.
Ad esempio, nel grafo in figura~\ref{fig:directed-graph-example}, le cricche sono
$\{\{v_1, v_2\}, \{v_3\}, \{v_4\}, \{v_5, v_6\}, \{v_7\}\}$. \newline
Si noti che l'insieme di nodi $V$ di un grafo completo \`e per definizione un'unica cricca. \newline

\`E altresì possibile definire il concetto di \textbf{connettività} al livello globale di un grafo: informalmente,
un grafo ha una tanto più elevata connettività quanti più sono gli archi che devono essere rimossi per disconnettere
il grafo in più sottografi isolati.

\subsection{Visita in profondit\`a}
La visita in profondit\`a di un grafo (in inglese \textit{depth-first search} o \textit{DFS}) \`e un particolare
algoritmo di visita che, in quanto tale, permette di visitare tutti i nodi di un grafo partendo da uno o più nodi
iniziali, scoprendo tutti i nodi raggiungibili da essi.
Ci\`o viene fatto in modo ricorsivo: nel corso della visita di un nodo si considerano uno ad uno i nodi ad esso
adiacenti, e ai nodi non ancora marcati come visitati viene applicata immediatamente la stessa procedura di visita.
\newpage

\begin{algorithm}[H]
    \caption{DFS($G$)}\label{alg:dfs}
    \begin{algorithmic}[1]
        \For {$v \in G.V$}
            \State $v.color \coloneqq$ WHITE
            \State $u.\pi \coloneqq$ NIL
        \EndFor
        \State $time \coloneqq 0$
        \For{$v \in G.V$}
            \If {$v.color ==$ WHITE}
                \State $DFS-VISIT(G, v)$
            \EndIf
        \EndFor
    \end{algorithmic}
\end{algorithm}
\begin{algorithm}[H]
    \caption{DFS-VISIT($G$, $u$)}\label{alg:dfs-visit}
    \begin{algorithmic}[1]
        \State $u.color \coloneqq$ GRAY
        \State $time \coloneqq time + 1$
        \State $u.d \coloneqq time$
        \For {$v \in G.Adj[u]$}
            \If {$v.color ==$ WHITE}
                \State $v.\pi \coloneqq u$
                \State $DFS-VISIT(G, v)$
            \EndIf
        \EndFor
        \State $u.color \coloneqq$ BLACK
        \State $time \coloneqq time + 1$
        \State $u.f \coloneqq time$
    \end{algorithmic}
\end{algorithm}


Nel corso dell'algoritmo i nodi vengono colorati di tre colori: bianco, grigio e nero, ad indicare rispettivamente
che il nodo non \`e stato visitato, che \`e in fase di visita e che \`e stato visitato.
Questo permette di non incorrere in cicli infiniti di visita nel caso non si stia visitando un grafo aciclico.
Nel corso dell'algoritmo vengono anche assegnati ai nodi due valori interi: il tempo di scoperta $d$ e il tempo di
fine visita $f$, che permettono di determinare, rispettivamente, il momento in cui i nodi vengono scoperti e colorati
di grigio e il tempo in cui la visita di un nodo termina, colorandosi di nero.
Un attributo aggiuntivo, il predecessore $\pi$, permette di memorizzare il nodo da cui si \`e scoperto il nodo
corrente e, con esso, di ricostruire il cammino di visita sotto forma di albero, detto albero di visita in profondit\`a.

\begin{figure}
    \resizebox{!}{3.07cm}{
    \begin{tikzpicture}
        %comp 1
        \node[mynode](n1) at (0,0){1};
        \node[mynode](n2) at (1.2,-1){2};
        \node[mynode](n3) at (1.2, 1){4};
        \node[mynode](n4) at (2.4, 0){3};
        \draw[myarrow](n1)--(n2);
        \draw[myarrow](n3)--(n1);
        \draw[myarrow](n4)--(n3);
        \draw[myarrow](n2)--(n4);
        %comp 2
        \node[mynode](n5) at (4,1.2){5};
        \node[mynode](n6) at (4,0){6};
        \node[mynode](n7) at (4, -1.2){7};
        \draw[myarrow](n6) -- (n5);
        \draw[myarrow](n5) to[out=3,in=3] (n7);
        \draw[myarrow](n7) -- (n6);

        %comp edges
        \draw[myarrow](n3) -- (n5);
        \draw[myarrow](n2) -- (n7);
        %comp 3
        \node[mynode](n8) at (6, 2){8};
        %comp 4
        \node[mynode, label=above:{$1$}, fill={rgb:black,1;white,2}](n9) at (8, 2){9};
        \node[mynode](na) at (8.5, 0){10};
        \draw[myarrow](n9) -- (n8);
        \draw[myarrow](na) to[out=3,in=3] (n9);
        \draw[myarrow](n9) -- (na);
        \draw[myarrow](n8) -- (n5);
    \end{tikzpicture} \quad \quad
    \begin{tikzpicture}
        %comp 1
        % \node[mynode, fill=black](n1) at (0,0){\textcolor{white}{1}};
        \node[mynode](n1) at (0,0){1};
        \node[mynode](n2) at (1.2,-1){2};
        \node[mynode](n3) at (1.2, 1){4};
        \node[mynode](n4) at (2.4, 0){3};
        \draw[myarrow](n1)--(n2);
        \draw[myarrow](n3)--(n1);
        \draw[myarrow](n4)--(n3);
        \draw[myarrow](n2)--(n4);
        %comp 2
        \node[mynode](n5) at (4,1.2){5};
        \node[mynode](n6) at (4,0){6};
        \node[mynode](n7) at (4, -1.2){7};
        \draw[myarrow](n6) -- (n5);
        \draw[myarrow](n5) to[out=3,in=3] (n7);
        \draw[myarrow](n7) -- (n6);

        %comp edges
        \draw[myarrow](n3) -- (n5);
        \draw[myarrow](n2) -- (n7);
        %comp 3
        \node[mynode](n8) at (6, 2){8};
        %comp 4
        \node[mynode, label=above:{$1$}, fill={rgb:black,1;white,2}](n9) at (8, 2){9};
        \node[mynode, label=below:{$2$}, fill={rgb:black,1;white,2}](na) at (8.5, 0){10};
        \draw[myarrow](n9) -- (n8);
        \draw[myarrow](na) to[out=3,in=3] (n9);
        \draw[myarrow](n9) -- (na);
        \draw[myarrow](n8) -- (n5);
    \end{tikzpicture}
}
\resizebox{!}{3.07cm}{
    \begin{tikzpicture}
        %comp 1
        \node[mynode](n1) at (0,0){1};
        \node[mynode](n2) at (1.2,-1){2};
        \node[mynode](n3) at (1.2, 1){4};
        \node[mynode](n4) at (2.4, 0){3};
        \draw[myarrow](n1)--(n2);
        \draw[myarrow](n3)--(n1);
        \draw[myarrow](n4)--(n3);
        \draw[myarrow](n2)--(n4);
        %comp 2
        \node[mynode](n5) at (4,1.2){5};
        \node[mynode](n6) at (4,0){6};
        \node[mynode](n7) at (4, -1.2){7};
        \draw[myarrow](n6) -- (n5);
        \draw[myarrow](n5) to[out=3,in=3] (n7);
        \draw[myarrow](n7) -- (n6);

        %comp edges
        \draw[myarrow](n3) -- (n5);
        \draw[myarrow](n2) -- (n7);
        %comp 3
        \node[mynode](n8) at (6, 2){8};
        %comp 4
        \node[mynode, label=above:{$1$}, fill={rgb:black,1;white,2}](n9) at (8, 2){9};
        \node[mynode, label=below:{$2/3$}, fill=black](na) at (8.5, 0){\textcolor{white}{10}};
        \draw[myarrow](n9) -- (n8);
        \draw[myarrow](na) to[out=3,in=3] (n9);
        \draw[myarrow](n9) -- (na);
        \draw[myarrow](n8) -- (n5);
    \end{tikzpicture} \quad \quad
    \begin{tikzpicture}
        %comp 1
        \node[mynode](n1) at (0,0){1};
        \node[mynode](n2) at (1.2,-1){2};
        \node[mynode](n3) at (1.2, 1){4};
        \node[mynode](n4) at (2.4, 0){3};
        \draw[myarrow](n1)--(n2);
        \draw[myarrow](n3)--(n1);
        \draw[myarrow](n4)--(n3);
        \draw[myarrow](n2)--(n4);
        %comp 2
        \node[mynode](n5) at (4,1.2){5};
        \node[mynode](n6) at (4,0){6};
        \node[mynode](n7) at (4, -1.2){7};
        \draw[myarrow](n6) -- (n5);
        \draw[myarrow](n5) to[out=3,in=3] (n7);
        \draw[myarrow](n7) -- (n6);

        %comp edges
        \draw[myarrow](n3) -- (n5);
        \draw[myarrow](n2) -- (n7);
        %comp 3
        \node[mynode, label=above:{$4$}, fill={rgb:black,1;white,2}](n8) at (6, 2){8};
        %comp 4
        \node[mynode, label=above:{$1$}, fill={rgb:black,1;white,2}](n9) at (8, 2){9};
        \node[mynode, label=below:{$2/3$}, fill=black](na) at (8.5, 0){\textcolor{white}{10}};
        \draw[myarrow](n9) -- (n8);
        \draw[myarrow](na) to[out=3,in=3] (n9);
        \draw[myarrow](n9) -- (na);
        \draw[myarrow](n8) -- (n5);
    \end{tikzpicture}
}
\resizebox{!}{3.416cm}{
    \begin{tikzpicture}
        %comp 1

        \node[mynode](n1) at (0,0){1};
        \node[mynode](n2) at (1.2,-1){2};
        \node[mynode](n3) at (1.2, 1){4};
        \node[mynode](n4) at (2.4, 0){3};
        \draw[myarrow](n1)--(n2);
        \draw[myarrow](n3)--(n1);
        \draw[myarrow](n4)--(n3);
        \draw[myarrow](n2)--(n4);
        %comp 2
        \node[mynode, label=above:{$5$}, fill={rgb:black,1;white,2}](n5) at (4,1.2){5};
        \node[mynode](n6) at (4,0){6};
        \node[mynode](n7) at (4, -1.2){7};
        \draw[myarrow](n6) -- (n5);
        \draw[myarrow](n5) to[out=3,in=3] (n7);
        \draw[myarrow](n7) -- (n6);

        %comp edges
        \draw[myarrow](n3) -- (n5);
        \draw[myarrow](n2) -- (n7);
        %comp 3
        \node[mynode, label=above:{$4$}, fill={rgb:black,1;white,2}](n8) at (6, 2){8};
        %comp 4
        \node[mynode, label=above:{$1$}, fill={rgb:black,1;white,2}](n9) at (8, 2){9};
        \node[mynode, label=below:{$2/3$}, fill=black](na) at (8.5, 0){\textcolor{white}{10}};
        \draw[myarrow](n9) -- (n8);
        \draw[myarrow](na) to[out=3,in=3] (n9);
        \draw[myarrow](n9) -- (na);
        \draw[myarrow](n8) -- (n5);
    \end{tikzpicture} \quad \quad
    \begin{tikzpicture}
        %comp 1

        \node[mynode](n1) at (0,0){1};
        \node[mynode](n2) at (1.2,-1){2};
        \node[mynode](n3) at (1.2, 1){4};
        \node[mynode](n4) at (2.4, 0){3};
        \draw[myarrow](n1)--(n2);
        \draw[myarrow](n3)--(n1);
        \draw[myarrow](n4)--(n3);
        \draw[myarrow](n2)--(n4);
        %comp 2
        \node[mynode, label=above:{$5$}, fill={rgb:black,1;white,2}](n5) at (4,1.2){5};
        \node[mynode](n6) at (4,0){6};
        \node[mynode, label=below:{$6$}, fill={rgb:black,1;white,2}](n7) at (4, -1.2){7};
        \draw[myarrow](n6) -- (n5);
        \draw[myarrow](n5) to[out=3,in=3] (n7);
        \draw[myarrow](n7) -- (n6);

        %comp edges
        \draw[myarrow](n3) -- (n5);
        \draw[myarrow](n2) -- (n7);
        %comp 3
        \node[mynode, label=above:{$4$}, fill={rgb:black,1;white,2}](n8) at (6, 2){8};
        %comp 4
        \node[mynode, label=above:{$1$}, fill={rgb:black,1;white,2}](n9) at (8, 2){9};
        \node[mynode, label=below:{$2/3$}, fill=black](na) at (8.5, 0){\textcolor{white}{10}};
        \draw[myarrow](n9) -- (n8);
        \draw[myarrow](na) to[out=3,in=3] (n9);
        \draw[myarrow](n9) -- (na);
        \draw[myarrow](n8) -- (n5);
    \end{tikzpicture}
}
\resizebox{!}{3.416cm}{
    \begin{tikzpicture}
        %comp 1
        \node[mynode](n1) at (0,0){1};
        \node[mynode](n2) at (1.2,-1){2};
        \node[mynode](n3) at (1.2, 1){4};
        \node[mynode](n4) at (2.4, 0){3};
        \draw[myarrow](n1)--(n2);
        \draw[myarrow](n3)--(n1);
        \draw[myarrow](n4)--(n3);
        \draw[myarrow](n2)--(n4);
        %comp 2
        \node[mynode, label=above:{$5$}, fill={rgb:black,1;white,2}](n5) at (4,1.2){5};
        \node[mynode, label=right:{$7$}, fill={rgb:black,1;white,2}](n6) at (4,0){6};
        \node[mynode, label=below:{$6$}, fill={rgb:black,1;white,2}](n7) at (4, -1.2){7};
        \draw[myarrow](n6) -- (n5);
        \draw[myarrow](n5) to[out=3,in=3] (n7);
        \draw[myarrow](n7) -- (n6);

        %comp edges
        \draw[myarrow](n3) -- (n5);
        \draw[myarrow](n2) -- (n7);
        %comp 3
        \node[mynode, label=above:{$4$}, fill={rgb:black,1;white,2}](n8) at (6, 2){8};
        %comp 4
        \node[mynode, label=above:{$1$}, fill={rgb:black,1;white,2}](n9) at (8, 2){9};
        \node[mynode, label=below:{$2/3$}, fill=black](na) at (8.5, 0){\textcolor{white}{10}};
        \draw[myarrow](n9) -- (n8);
        \draw[myarrow](na) to[out=3,in=3] (n9);
        \draw[myarrow](n9) -- (na);
        \draw[myarrow](n8) -- (n5);
    \end{tikzpicture} \quad \quad
    \begin{tikzpicture}
        %comp 1
        \node[mynode](n1) at (0,0){1};
        \node[mynode](n2) at (1.2,-1){2};
        \node[mynode](n3) at (1.2, 1){4};
        \node[mynode](n4) at (2.4, 0){3};
        \draw[myarrow](n1)--(n2);
        \draw[myarrow](n3)--(n1);
        \draw[myarrow](n4)--(n3);
        \draw[myarrow](n2)--(n4);
        %comp 2
        \node[mynode, label=above:{$5$}, fill={rgb:black,1;white,2}](n5) at (4,1.2){5};
        \node[mynode, label=right:{$7/8$}, fill=black](n6) at (4,0){\textcolor{white}{6}};
        \node[mynode, label=below:{$6$}, fill={rgb:black,1;white,2}](n7) at (4, -1.2){7};
        \draw[myarrow](n6) -- (n5);
        \draw[myarrow](n5) to[out=3,in=3] (n7);
        \draw[myarrow](n7) -- (n6);

        %comp edges
        \draw[myarrow](n3) -- (n5);
        \draw[myarrow](n2) -- (n7);
        %comp 3
        \node[mynode, label=above:{$4$}, fill={rgb:black,1;white,2}](n8) at (6, 2){8};
        %comp 4
        \node[mynode, label=above:{$1$}, fill={rgb:black,1;white,2}](n9) at (8, 2){9};
        \node[mynode, label=below:{$2/3$}, fill=black](na) at (8.5, 0){\textcolor{white}{10}};
        \draw[myarrow](n9) -- (n8);
        \draw[myarrow](na) to[out=3,in=3] (n9);
        \draw[myarrow](n9) -- (na);
        \draw[myarrow](n8) -- (n5);
    \end{tikzpicture}
}
\resizebox{!}{3.5cm}{
    \begin{tikzpicture}
        %comp 1
        \node[mynode](n1) at (0,0){1};
        \node[mynode](n2) at (1.2,-1){2};
        \node[mynode](n3) at (1.2, 1){4};
        \node[mynode](n4) at (2.4, 0){3};
        \draw[myarrow](n1)--(n2);
        \draw[myarrow](n3)--(n1);
        \draw[myarrow](n4)--(n3);
        \draw[myarrow](n2)--(n4);
        %comp 2
        \node[mynode, label=above:{$5$}, fill={rgb:black,1;white,2}](n5) at (4,1.2){5};
        \node[mynode, label=right:{$7/8$}, fill=black](n6) at (4,0){\textcolor{white}{6}};
        \node[mynode, label=below:{$6/9$}, fill=black](n7) at (4, -1.2){\textcolor{white}{7}};
        \draw[myarrow](n6) -- (n5);
        \draw[myarrow](n5) to[out=3,in=3] (n7);
        \draw[myarrow](n7) -- (n6);

        %comp edges
        \draw[myarrow](n3) -- (n5);
        \draw[myarrow](n2) -- (n7);
        %comp 3
        \node[mynode, label=above:{$4$}, fill={rgb:black,1;white,2}](n8) at (6, 2){8};
        %comp 4
        \node[mynode, label=above:{$1$}, fill={rgb:black,1;white,2}](n9) at (8, 2){9};
        \node[mynode, label=below:{$2/3$}, fill=black](na) at (8.5, 0){\textcolor{white}{10}};
        \draw[myarrow](n9) -- (n8);
        \draw[myarrow](na) to[out=3,in=3] (n9);
        \draw[myarrow](n9) -- (na);
        \draw[myarrow](n8) -- (n5);
    \end{tikzpicture} \quad \quad
    \begin{tikzpicture}
        % comp 1
        \node[mynode](n1) at (0,0){1};
        \node[mynode](n2) at (1.2,-1){2};
        \node[mynode](n3) at (1.2, 1){4};
        \node[mynode](n4) at (2.4, 0){3};
        \draw[myarrow](n1)--(n2);
        \draw[myarrow](n3)--(n1);
        \draw[myarrow](n4)--(n3);
        \draw[myarrow](n2)--(n4);
        %comp 2
        \node[mynode, label=above:{$5/10$}, fill=black](n5) at (4,1.2){\textcolor{white}{5}};
        \node[mynode, label=right:{$7/8$}, fill=black](n6) at (4,0){\textcolor{white}{6}};
        \node[mynode, label=below:{$6/9$}, fill=black](n7) at (4, -1.2){\textcolor{white}{7}};
        \draw[myarrow](n6) -- (n5);
        \draw[myarrow](n5) to[out=3,in=3] (n7);
        \draw[myarrow](n7) -- (n6);

        %comp edges
        \draw[myarrow](n3) -- (n5);
        \draw[myarrow](n2) -- (n7);
        %comp 3
        \node[mynode, label=above:{$4$}, fill={rgb:black,1;white,2}](n8) at (6, 2){8};
        %comp 4
        \node[mynode, label=above:{$1$}, fill={rgb:black,1;white,2}](n9) at (8, 2){9};
        \node[draw, thick, circle, label=below:{$2/3$}, fill=black](na) at (8.5, 0){\textcolor{white}{10}};
        \draw[myarrow](n9) -- (n8);
        \draw[myarrow](na) to[out=3,in=3] (n9);
        \draw[myarrow](n9) -- (na);
        \draw[myarrow](n8) -- (n5);
    \end{tikzpicture}
}
\resizebox{!}{3.59cm}{
    \begin{tikzpicture}
        %comp 1
        % \node[mynode, fill=black](n1) at (0,0){\textcolor{white}{1}};
        \node[mynode](n1) at (0,0){1};
        \node[mynode](n2) at (1.2,-1){2};
        \node[mynode](n3) at (1.2, 1){4};
        \node[mynode](n4) at (2.4, 0){3};
        \draw[myarrow](n1)--(n2);
        \draw[myarrow](n3)--(n1);
        \draw[myarrow](n4)--(n3);
        \draw[myarrow](n2)--(n4);
        %comp 2
        \node[mynode, label=above:{$5/10$}, fill=black](n5) at (4,1.2){\textcolor{white}{5}};
        \node[mynode, label=right:{$7/8$}, fill=black](n6) at (4,0){\textcolor{white}{6}};
        \node[mynode, label=below:{$6/9$}, fill=black](n7) at (4, -1.2){\textcolor{white}{7}};
        \draw[myarrow](n6) -- (n5);
        \draw[myarrow](n5) to[out=3,in=3] (n7);
        \draw[myarrow](n7) -- (n6);

        %comp edges
        \draw[myarrow](n3) -- (n5);
        \draw[myarrow](n2) -- (n7);
        %comp 3
        \node[mynode, label=above:{$4/11$}, fill=black](n8) at (6, 2){\textcolor{white}{8}};
        %comp 4
        \node[mynode, label=above:{$1$}, fill={rgb:black,1;white,2}](n9) at (8, 2){9};
        \node[mynode, label=below:{$2/3$}, fill=black](na) at (8.5, 0){\textcolor{white}{10}};
        \draw[myarrow](n9) -- (n8);
        \draw[myarrow](na) to[out=3,in=3] (n9);
        \draw[myarrow](n9) -- (na);
        \draw[myarrow](n8) -- (n5);
    \end{tikzpicture} \quad \quad
    \begin{tikzpicture}
        %comp 1
        % \node[mynode, fill=black](n1) at (0,0){\textcolor{white}{1}};
        \node[mynode](n1) at (0,0){1};
        \node[mynode](n2) at (1.2,-1){2};
        \node[mynode](n3) at (1.2, 1){4};
        \node[mynode](n4) at (2.4, 0){3};
        \draw[myarrow](n1)--(n2);
        \draw[myarrow](n3)--(n1);
        \draw[myarrow](n4)--(n3);
        \draw[myarrow](n2)--(n4);
        %comp 2
        \node[mynode, label=above:{$5/10$}, fill=black](n5) at (4,1.2){\textcolor{white}{5}};
        \node[mynode, label=right:{$7/8$}, fill=black](n6) at (4,0){\textcolor{white}{6}};
        \node[mynode, label=below:{$6/9$}, fill=black](n7) at (4, -1.2){\textcolor{white}{7}};
        \draw[myarrow](n6) -- (n5);
        \draw[myarrow](n5) to[out=3,in=3] (n7);
        \draw[myarrow](n7) -- (n6);

        %comp edges
        \draw[myarrow](n3) -- (n5);
        \draw[myarrow](n2) -- (n7);
        %comp 3
        \node[mynode, label=above:{$4/11$}, fill=black](n8) at (6, 2){\textcolor{white}{8}};
        %comp 4
        \node[mynode, label=above:{$1/12$}, fill=black](n9) at (8, 2){\textcolor{white}{9}};
        \node[mynode, label=below:{$2/3$}, fill=black](na) at (8.5, 0){\textcolor{white}{10}};
        \draw[myarrow](n9) -- (n8);
        \draw[myarrow](na) to[out=3,in=3] (n9);
        \draw[myarrow](n9) -- (na);
        \draw[myarrow](n8) -- (n5);
    \end{tikzpicture}
}

    \caption{Esempio di esecuzione di una visita in profondit\`a su un grafo diretto}
    \label{fig:dfs-example}
\end{figure}

In figura~\ref{fig:dfs-example} \`e rappresentato un esempio di esecuzione della procedura DFS-VISIT su un grafo
diretto, eseguita a partire dal nodo con etichetta $9$, in cui lo stato del grafo \`e rappresentato ad ogni
incremento della variabile $time$.

\paragraph{Complessità}
I cicli for alle righe 1 e 6 nello pseudocodice comportano un tempo $\Theta(|V|)$ se non tiene conto della chiamata a
DFS-VISIT. Tuttavia, attraverso il metodo dell'aggregazione, si può affermare che la totalità delle chiamate alla
procedura ricorsiva DFS-VISIT comportano un tempo $\Theta(|V| + |E|)$, in quanto ogni nodo viene visitato una sola
volta, quando è ancora bianco, e ogni arco viene esaminato al massimo una volta, in quanto uscente da un solo
nodo.
L'intera procedura si mantiene, quindi, di tempo lineare rispetto al numero di nodi e di archi del grafo,
risultando esattamente $\mathbf{\Theta(|V| + |E|)}$.