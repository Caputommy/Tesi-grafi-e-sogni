\chapter{Algoritmi di enumerazione}

In informatica, un \textit{algoritmo di enumerazione} \`e un algoritmo che elenca tutte le possibili risposte ad un
problema computazionale in modo sistematico e completo. Tali algoritmi sono quindi progettati per ricevere un
determinato input, generare una lista esaustiva di tutte le possibili soluzioni senza duplicati e, solo allora,
terminare.
Quando si parla di algoritmi di enumerazione applicati al dominio dei grafi, allora spesso si intende il processo di
identificazione di sottoinsiemi di nodi o archi che soddisfano determinate caratteristiche.
In questo capitolo si tratteranno e analizzeranno alcuni utili algoritmi di enumerazione presenti in letteratura per
il riconoscimento di pattern strutturali all'interno di grafi. In particolare, si scuteranno algoritmi per l'enumerazione
di componenti fortemente connesse, di cricche e di circuiti semplici, problemi di notevole importanza nella teoria dei
grafi e ritenuti di interesse per la definizione di algoritmi di contrazione.
Questi algoritmi costituiranno il \"motore\" degli algoritmi presentati nei capitoli successivi, e saranno di
importanza fondamentale alla definizione degli algoritmi di contrazione usati per la costruzione di grafi multi-livello.

\section{Enumerazione di componenti fortemente connesse}\label{subsec:enumerazione-di-componenti-fortemente-connesse}
Come citato nel Capitolo 1, la condensazione di un grafo diretto \`e il suo grafo quoziente dove le componenti
fortemente connesse definiscono i blocchi della partizione.
Enumerare le componenti fortemente connesse di un grafo diretto $G = (V, E)$ significa trovare tutti i sottoinsiemi
che permettano di costruire una condensazione del grafo di partenza.
In questa sezione si discuter\`a un classico algoritmo di enumerazione delle componenti fortementi connesse, chiamato
da alcuni testi come l'algoritmo di Kosaraju~\cite{SHARIR198167} e al seguito si discuter\`a di un possibile adattamento
dell'algoritmo per la costruzione contestuale di un grafo contratto in forma di grafo decontraibile.

\subsection{Algoritmo di Kosaraju}\label{subsec:algoritmo-di-kosaraju}
L'algoritmo di Kosaraju (anche noto come algoritmo di Kosaraju-Sharir) \`e un algoritmo per l'enumerazione
delle componenti fortemente connesse di un grafo diretto dalla complessit\`a lineare scoperto nel 1978 da S. Rao
Kosaraju, ma pubblicato solamente nel 1981 da Micha Sharir, che lo scopri\`o indipendentemente.
Esso sfrutta il principio per cui le componenti fortemente connesse di un grafo diretto sono le stesse del suo grafo
trasposto, ovvero il grafo ottenuto invertendo l'orientamento di tutti gli archi. \newline

A seguire alcune utili nozioni preliminari per la comprensione dell'algoritmo:

\paragraph{Grafo trasposto}
Dato un grafo diretto $G = (V, E)$, il suo grafo trasposto $G^T = (V, E^T)$ \`e il grafo ottenuto invertendo
l'orientamento di tutti gli archi di $G$, ovvero $E^T = \{(u, v) \mid (v, u) \in E\}$.
 \'E interessante notare che $G^T$ ha le stesse componenti fortemente connesse di $G$: se un cammino da $u$ a $v$
esiste in $G$, allora esiste anche un cammino da $v$ a $u$ in $G^T$.
Essendo le componenti fortemente connesse basate sulla mutua raggiungibilit\`a dei nodi, esse non cambiano
quando si invertono gli archi.

Una procedura algoritmica per il calcolo di $G^T$ non farebbe altro che scorrere linsieme di archi di $G$ e
invertirli, e sarebbe quindi di complessit\`a lineare.

\paragraph{Visita in profondit\`a}
La visita in profondit\`a di un grafo (in inglese \textit{depth-first search} o \textit{DFS}) \`e un particolare
algoritmo di visita che, in quanto tale, permette di visitare tutti i nodi di un grafo partendo da un nodo iniziale,
e di scoprire tutti i nodi raggiungibili da esso.
Ci\`o viene fatto in modo ricorsivo: nel corso della visita di un nodo si considerano uno ad uno i nodi ad esso
adiacenti, e ai nodi non ancora marcati come visitati viene applicata immediatamente la stessa procedura di visita.

\begin{algorithm}[H]
    \caption{DFS($G$)}\label{alg:dfs}
    \begin{algorithmic}[1]
        \For {$v \in G.V$}
            \State $v.color \coloneqq$ WHITE
            \State $u.\pi \coloneqq$ NIL
        \EndFor
        \State $time \coloneqq 0$
        \For{$v \in G.V$}
            \If {$v.color ==$ WHITE}
                \State $DFS-VISIT(G, v)$
            \EndIf
        \EndFor
    \end{algorithmic}
\end{algorithm}
\begin{algorithm}[H]
    \caption{DFS-VISIT($G$, $u$)}\label{alg:dfs-visit}
    \begin{algorithmic}[1]
        \State $u.color \coloneqq$ GRAY
        \State $time \coloneqq time + 1$
        \State $u.d \coloneqq time$
        \For {$v \in G.Adj[u]$}
            \If {$v.color ==$ WHITE}
                \State $v.\pi \coloneqq u$
                \State $DFS-VISIT(G, v)$
            \EndIf
        \EndFor
        \State $u.color \coloneqq$ BLACK
        \State $time \coloneqq time + 1$
        \State $u.f \coloneqq time$
    \end{algorithmic}
\end{algorithm}


\begin{figure}
    \resizebox{!}{3.07cm}{
    \begin{tikzpicture}
        %comp 1
        \node[mynode](n1) at (0,0){1};
        \node[mynode](n2) at (1.2,-1){2};
        \node[mynode](n3) at (1.2, 1){4};
        \node[mynode](n4) at (2.4, 0){3};
        \draw[myarrow](n1)--(n2);
        \draw[myarrow](n3)--(n1);
        \draw[myarrow](n4)--(n3);
        \draw[myarrow](n2)--(n4);
        %comp 2
        \node[mynode](n5) at (4,1.2){5};
        \node[mynode](n6) at (4,0){6};
        \node[mynode](n7) at (4, -1.2){7};
        \draw[myarrow](n6) -- (n5);
        \draw[myarrow](n5) to[out=3,in=3] (n7);
        \draw[myarrow](n7) -- (n6);

        %comp edges
        \draw[myarrow](n3) -- (n5);
        \draw[myarrow](n2) -- (n7);
        %comp 3
        \node[mynode](n8) at (6, 2){8};
        %comp 4
        \node[mynode, label=above:{$1$}, fill={rgb:black,1;white,2}](n9) at (8, 2){9};
        \node[mynode](na) at (8.5, 0){10};
        \draw[myarrow](n9) -- (n8);
        \draw[myarrow](na) to[out=3,in=3] (n9);
        \draw[myarrow](n9) -- (na);
        \draw[myarrow](n8) -- (n5);
    \end{tikzpicture} \quad \quad
    \begin{tikzpicture}
        %comp 1
        % \node[mynode, fill=black](n1) at (0,0){\textcolor{white}{1}};
        \node[mynode](n1) at (0,0){1};
        \node[mynode](n2) at (1.2,-1){2};
        \node[mynode](n3) at (1.2, 1){4};
        \node[mynode](n4) at (2.4, 0){3};
        \draw[myarrow](n1)--(n2);
        \draw[myarrow](n3)--(n1);
        \draw[myarrow](n4)--(n3);
        \draw[myarrow](n2)--(n4);
        %comp 2
        \node[mynode](n5) at (4,1.2){5};
        \node[mynode](n6) at (4,0){6};
        \node[mynode](n7) at (4, -1.2){7};
        \draw[myarrow](n6) -- (n5);
        \draw[myarrow](n5) to[out=3,in=3] (n7);
        \draw[myarrow](n7) -- (n6);

        %comp edges
        \draw[myarrow](n3) -- (n5);
        \draw[myarrow](n2) -- (n7);
        %comp 3
        \node[mynode](n8) at (6, 2){8};
        %comp 4
        \node[mynode, label=above:{$1$}, fill={rgb:black,1;white,2}](n9) at (8, 2){9};
        \node[mynode, label=below:{$2$}, fill={rgb:black,1;white,2}](na) at (8.5, 0){10};
        \draw[myarrow](n9) -- (n8);
        \draw[myarrow](na) to[out=3,in=3] (n9);
        \draw[myarrow](n9) -- (na);
        \draw[myarrow](n8) -- (n5);
    \end{tikzpicture}
}
\resizebox{!}{3.07cm}{
    \begin{tikzpicture}
        %comp 1
        \node[mynode](n1) at (0,0){1};
        \node[mynode](n2) at (1.2,-1){2};
        \node[mynode](n3) at (1.2, 1){4};
        \node[mynode](n4) at (2.4, 0){3};
        \draw[myarrow](n1)--(n2);
        \draw[myarrow](n3)--(n1);
        \draw[myarrow](n4)--(n3);
        \draw[myarrow](n2)--(n4);
        %comp 2
        \node[mynode](n5) at (4,1.2){5};
        \node[mynode](n6) at (4,0){6};
        \node[mynode](n7) at (4, -1.2){7};
        \draw[myarrow](n6) -- (n5);
        \draw[myarrow](n5) to[out=3,in=3] (n7);
        \draw[myarrow](n7) -- (n6);

        %comp edges
        \draw[myarrow](n3) -- (n5);
        \draw[myarrow](n2) -- (n7);
        %comp 3
        \node[mynode](n8) at (6, 2){8};
        %comp 4
        \node[mynode, label=above:{$1$}, fill={rgb:black,1;white,2}](n9) at (8, 2){9};
        \node[mynode, label=below:{$2/3$}, fill=black](na) at (8.5, 0){\textcolor{white}{10}};
        \draw[myarrow](n9) -- (n8);
        \draw[myarrow](na) to[out=3,in=3] (n9);
        \draw[myarrow](n9) -- (na);
        \draw[myarrow](n8) -- (n5);
    \end{tikzpicture} \quad \quad
    \begin{tikzpicture}
        %comp 1
        \node[mynode](n1) at (0,0){1};
        \node[mynode](n2) at (1.2,-1){2};
        \node[mynode](n3) at (1.2, 1){4};
        \node[mynode](n4) at (2.4, 0){3};
        \draw[myarrow](n1)--(n2);
        \draw[myarrow](n3)--(n1);
        \draw[myarrow](n4)--(n3);
        \draw[myarrow](n2)--(n4);
        %comp 2
        \node[mynode](n5) at (4,1.2){5};
        \node[mynode](n6) at (4,0){6};
        \node[mynode](n7) at (4, -1.2){7};
        \draw[myarrow](n6) -- (n5);
        \draw[myarrow](n5) to[out=3,in=3] (n7);
        \draw[myarrow](n7) -- (n6);

        %comp edges
        \draw[myarrow](n3) -- (n5);
        \draw[myarrow](n2) -- (n7);
        %comp 3
        \node[mynode, label=above:{$4$}, fill={rgb:black,1;white,2}](n8) at (6, 2){8};
        %comp 4
        \node[mynode, label=above:{$1$}, fill={rgb:black,1;white,2}](n9) at (8, 2){9};
        \node[mynode, label=below:{$2/3$}, fill=black](na) at (8.5, 0){\textcolor{white}{10}};
        \draw[myarrow](n9) -- (n8);
        \draw[myarrow](na) to[out=3,in=3] (n9);
        \draw[myarrow](n9) -- (na);
        \draw[myarrow](n8) -- (n5);
    \end{tikzpicture}
}
\resizebox{!}{3.416cm}{
    \begin{tikzpicture}
        %comp 1

        \node[mynode](n1) at (0,0){1};
        \node[mynode](n2) at (1.2,-1){2};
        \node[mynode](n3) at (1.2, 1){4};
        \node[mynode](n4) at (2.4, 0){3};
        \draw[myarrow](n1)--(n2);
        \draw[myarrow](n3)--(n1);
        \draw[myarrow](n4)--(n3);
        \draw[myarrow](n2)--(n4);
        %comp 2
        \node[mynode, label=above:{$5$}, fill={rgb:black,1;white,2}](n5) at (4,1.2){5};
        \node[mynode](n6) at (4,0){6};
        \node[mynode](n7) at (4, -1.2){7};
        \draw[myarrow](n6) -- (n5);
        \draw[myarrow](n5) to[out=3,in=3] (n7);
        \draw[myarrow](n7) -- (n6);

        %comp edges
        \draw[myarrow](n3) -- (n5);
        \draw[myarrow](n2) -- (n7);
        %comp 3
        \node[mynode, label=above:{$4$}, fill={rgb:black,1;white,2}](n8) at (6, 2){8};
        %comp 4
        \node[mynode, label=above:{$1$}, fill={rgb:black,1;white,2}](n9) at (8, 2){9};
        \node[mynode, label=below:{$2/3$}, fill=black](na) at (8.5, 0){\textcolor{white}{10}};
        \draw[myarrow](n9) -- (n8);
        \draw[myarrow](na) to[out=3,in=3] (n9);
        \draw[myarrow](n9) -- (na);
        \draw[myarrow](n8) -- (n5);
    \end{tikzpicture} \quad \quad
    \begin{tikzpicture}
        %comp 1

        \node[mynode](n1) at (0,0){1};
        \node[mynode](n2) at (1.2,-1){2};
        \node[mynode](n3) at (1.2, 1){4};
        \node[mynode](n4) at (2.4, 0){3};
        \draw[myarrow](n1)--(n2);
        \draw[myarrow](n3)--(n1);
        \draw[myarrow](n4)--(n3);
        \draw[myarrow](n2)--(n4);
        %comp 2
        \node[mynode, label=above:{$5$}, fill={rgb:black,1;white,2}](n5) at (4,1.2){5};
        \node[mynode](n6) at (4,0){6};
        \node[mynode, label=below:{$6$}, fill={rgb:black,1;white,2}](n7) at (4, -1.2){7};
        \draw[myarrow](n6) -- (n5);
        \draw[myarrow](n5) to[out=3,in=3] (n7);
        \draw[myarrow](n7) -- (n6);

        %comp edges
        \draw[myarrow](n3) -- (n5);
        \draw[myarrow](n2) -- (n7);
        %comp 3
        \node[mynode, label=above:{$4$}, fill={rgb:black,1;white,2}](n8) at (6, 2){8};
        %comp 4
        \node[mynode, label=above:{$1$}, fill={rgb:black,1;white,2}](n9) at (8, 2){9};
        \node[mynode, label=below:{$2/3$}, fill=black](na) at (8.5, 0){\textcolor{white}{10}};
        \draw[myarrow](n9) -- (n8);
        \draw[myarrow](na) to[out=3,in=3] (n9);
        \draw[myarrow](n9) -- (na);
        \draw[myarrow](n8) -- (n5);
    \end{tikzpicture}
}
\resizebox{!}{3.416cm}{
    \begin{tikzpicture}
        %comp 1
        \node[mynode](n1) at (0,0){1};
        \node[mynode](n2) at (1.2,-1){2};
        \node[mynode](n3) at (1.2, 1){4};
        \node[mynode](n4) at (2.4, 0){3};
        \draw[myarrow](n1)--(n2);
        \draw[myarrow](n3)--(n1);
        \draw[myarrow](n4)--(n3);
        \draw[myarrow](n2)--(n4);
        %comp 2
        \node[mynode, label=above:{$5$}, fill={rgb:black,1;white,2}](n5) at (4,1.2){5};
        \node[mynode, label=right:{$7$}, fill={rgb:black,1;white,2}](n6) at (4,0){6};
        \node[mynode, label=below:{$6$}, fill={rgb:black,1;white,2}](n7) at (4, -1.2){7};
        \draw[myarrow](n6) -- (n5);
        \draw[myarrow](n5) to[out=3,in=3] (n7);
        \draw[myarrow](n7) -- (n6);

        %comp edges
        \draw[myarrow](n3) -- (n5);
        \draw[myarrow](n2) -- (n7);
        %comp 3
        \node[mynode, label=above:{$4$}, fill={rgb:black,1;white,2}](n8) at (6, 2){8};
        %comp 4
        \node[mynode, label=above:{$1$}, fill={rgb:black,1;white,2}](n9) at (8, 2){9};
        \node[mynode, label=below:{$2/3$}, fill=black](na) at (8.5, 0){\textcolor{white}{10}};
        \draw[myarrow](n9) -- (n8);
        \draw[myarrow](na) to[out=3,in=3] (n9);
        \draw[myarrow](n9) -- (na);
        \draw[myarrow](n8) -- (n5);
    \end{tikzpicture} \quad \quad
    \begin{tikzpicture}
        %comp 1
        \node[mynode](n1) at (0,0){1};
        \node[mynode](n2) at (1.2,-1){2};
        \node[mynode](n3) at (1.2, 1){4};
        \node[mynode](n4) at (2.4, 0){3};
        \draw[myarrow](n1)--(n2);
        \draw[myarrow](n3)--(n1);
        \draw[myarrow](n4)--(n3);
        \draw[myarrow](n2)--(n4);
        %comp 2
        \node[mynode, label=above:{$5$}, fill={rgb:black,1;white,2}](n5) at (4,1.2){5};
        \node[mynode, label=right:{$7/8$}, fill=black](n6) at (4,0){\textcolor{white}{6}};
        \node[mynode, label=below:{$6$}, fill={rgb:black,1;white,2}](n7) at (4, -1.2){7};
        \draw[myarrow](n6) -- (n5);
        \draw[myarrow](n5) to[out=3,in=3] (n7);
        \draw[myarrow](n7) -- (n6);

        %comp edges
        \draw[myarrow](n3) -- (n5);
        \draw[myarrow](n2) -- (n7);
        %comp 3
        \node[mynode, label=above:{$4$}, fill={rgb:black,1;white,2}](n8) at (6, 2){8};
        %comp 4
        \node[mynode, label=above:{$1$}, fill={rgb:black,1;white,2}](n9) at (8, 2){9};
        \node[mynode, label=below:{$2/3$}, fill=black](na) at (8.5, 0){\textcolor{white}{10}};
        \draw[myarrow](n9) -- (n8);
        \draw[myarrow](na) to[out=3,in=3] (n9);
        \draw[myarrow](n9) -- (na);
        \draw[myarrow](n8) -- (n5);
    \end{tikzpicture}
}
\resizebox{!}{3.5cm}{
    \begin{tikzpicture}
        %comp 1
        \node[mynode](n1) at (0,0){1};
        \node[mynode](n2) at (1.2,-1){2};
        \node[mynode](n3) at (1.2, 1){4};
        \node[mynode](n4) at (2.4, 0){3};
        \draw[myarrow](n1)--(n2);
        \draw[myarrow](n3)--(n1);
        \draw[myarrow](n4)--(n3);
        \draw[myarrow](n2)--(n4);
        %comp 2
        \node[mynode, label=above:{$5$}, fill={rgb:black,1;white,2}](n5) at (4,1.2){5};
        \node[mynode, label=right:{$7/8$}, fill=black](n6) at (4,0){\textcolor{white}{6}};
        \node[mynode, label=below:{$6/9$}, fill=black](n7) at (4, -1.2){\textcolor{white}{7}};
        \draw[myarrow](n6) -- (n5);
        \draw[myarrow](n5) to[out=3,in=3] (n7);
        \draw[myarrow](n7) -- (n6);

        %comp edges
        \draw[myarrow](n3) -- (n5);
        \draw[myarrow](n2) -- (n7);
        %comp 3
        \node[mynode, label=above:{$4$}, fill={rgb:black,1;white,2}](n8) at (6, 2){8};
        %comp 4
        \node[mynode, label=above:{$1$}, fill={rgb:black,1;white,2}](n9) at (8, 2){9};
        \node[mynode, label=below:{$2/3$}, fill=black](na) at (8.5, 0){\textcolor{white}{10}};
        \draw[myarrow](n9) -- (n8);
        \draw[myarrow](na) to[out=3,in=3] (n9);
        \draw[myarrow](n9) -- (na);
        \draw[myarrow](n8) -- (n5);
    \end{tikzpicture} \quad \quad
    \begin{tikzpicture}
        % comp 1
        \node[mynode](n1) at (0,0){1};
        \node[mynode](n2) at (1.2,-1){2};
        \node[mynode](n3) at (1.2, 1){4};
        \node[mynode](n4) at (2.4, 0){3};
        \draw[myarrow](n1)--(n2);
        \draw[myarrow](n3)--(n1);
        \draw[myarrow](n4)--(n3);
        \draw[myarrow](n2)--(n4);
        %comp 2
        \node[mynode, label=above:{$5/10$}, fill=black](n5) at (4,1.2){\textcolor{white}{5}};
        \node[mynode, label=right:{$7/8$}, fill=black](n6) at (4,0){\textcolor{white}{6}};
        \node[mynode, label=below:{$6/9$}, fill=black](n7) at (4, -1.2){\textcolor{white}{7}};
        \draw[myarrow](n6) -- (n5);
        \draw[myarrow](n5) to[out=3,in=3] (n7);
        \draw[myarrow](n7) -- (n6);

        %comp edges
        \draw[myarrow](n3) -- (n5);
        \draw[myarrow](n2) -- (n7);
        %comp 3
        \node[mynode, label=above:{$4$}, fill={rgb:black,1;white,2}](n8) at (6, 2){8};
        %comp 4
        \node[mynode, label=above:{$1$}, fill={rgb:black,1;white,2}](n9) at (8, 2){9};
        \node[draw, thick, circle, label=below:{$2/3$}, fill=black](na) at (8.5, 0){\textcolor{white}{10}};
        \draw[myarrow](n9) -- (n8);
        \draw[myarrow](na) to[out=3,in=3] (n9);
        \draw[myarrow](n9) -- (na);
        \draw[myarrow](n8) -- (n5);
    \end{tikzpicture}
}
\resizebox{!}{3.59cm}{
    \begin{tikzpicture}
        %comp 1
        % \node[mynode, fill=black](n1) at (0,0){\textcolor{white}{1}};
        \node[mynode](n1) at (0,0){1};
        \node[mynode](n2) at (1.2,-1){2};
        \node[mynode](n3) at (1.2, 1){4};
        \node[mynode](n4) at (2.4, 0){3};
        \draw[myarrow](n1)--(n2);
        \draw[myarrow](n3)--(n1);
        \draw[myarrow](n4)--(n3);
        \draw[myarrow](n2)--(n4);
        %comp 2
        \node[mynode, label=above:{$5/10$}, fill=black](n5) at (4,1.2){\textcolor{white}{5}};
        \node[mynode, label=right:{$7/8$}, fill=black](n6) at (4,0){\textcolor{white}{6}};
        \node[mynode, label=below:{$6/9$}, fill=black](n7) at (4, -1.2){\textcolor{white}{7}};
        \draw[myarrow](n6) -- (n5);
        \draw[myarrow](n5) to[out=3,in=3] (n7);
        \draw[myarrow](n7) -- (n6);

        %comp edges
        \draw[myarrow](n3) -- (n5);
        \draw[myarrow](n2) -- (n7);
        %comp 3
        \node[mynode, label=above:{$4/11$}, fill=black](n8) at (6, 2){\textcolor{white}{8}};
        %comp 4
        \node[mynode, label=above:{$1$}, fill={rgb:black,1;white,2}](n9) at (8, 2){9};
        \node[mynode, label=below:{$2/3$}, fill=black](na) at (8.5, 0){\textcolor{white}{10}};
        \draw[myarrow](n9) -- (n8);
        \draw[myarrow](na) to[out=3,in=3] (n9);
        \draw[myarrow](n9) -- (na);
        \draw[myarrow](n8) -- (n5);
    \end{tikzpicture} \quad \quad
    \begin{tikzpicture}
        %comp 1
        % \node[mynode, fill=black](n1) at (0,0){\textcolor{white}{1}};
        \node[mynode](n1) at (0,0){1};
        \node[mynode](n2) at (1.2,-1){2};
        \node[mynode](n3) at (1.2, 1){4};
        \node[mynode](n4) at (2.4, 0){3};
        \draw[myarrow](n1)--(n2);
        \draw[myarrow](n3)--(n1);
        \draw[myarrow](n4)--(n3);
        \draw[myarrow](n2)--(n4);
        %comp 2
        \node[mynode, label=above:{$5/10$}, fill=black](n5) at (4,1.2){\textcolor{white}{5}};
        \node[mynode, label=right:{$7/8$}, fill=black](n6) at (4,0){\textcolor{white}{6}};
        \node[mynode, label=below:{$6/9$}, fill=black](n7) at (4, -1.2){\textcolor{white}{7}};
        \draw[myarrow](n6) -- (n5);
        \draw[myarrow](n5) to[out=3,in=3] (n7);
        \draw[myarrow](n7) -- (n6);

        %comp edges
        \draw[myarrow](n3) -- (n5);
        \draw[myarrow](n2) -- (n7);
        %comp 3
        \node[mynode, label=above:{$4/11$}, fill=black](n8) at (6, 2){\textcolor{white}{8}};
        %comp 4
        \node[mynode, label=above:{$1/12$}, fill=black](n9) at (8, 2){\textcolor{white}{9}};
        \node[mynode, label=below:{$2/3$}, fill=black](na) at (8.5, 0){\textcolor{white}{10}};
        \draw[myarrow](n9) -- (n8);
        \draw[myarrow](na) to[out=3,in=3] (n9);
        \draw[myarrow](n9) -- (na);
        \draw[myarrow](n8) -- (n5);
    \end{tikzpicture}
}

    \caption{Esempio di esecuzione di una visita in profondit\`a su un grafo diretto}
    \label{fig:dfs-example}
\end{figure}

Nel corso dell'algoritmo i nodi vengono colorati in tre colori: bianco, grigio e nero, ad indicare rispettivamente
che il nodo non \`e stato visitato, che \`e in fase di visita e che \`e stato visitato.
Questo permette di non incorrere in cicli di visita infiniti nel caso non si stia visitando un grafo aciclico.
Nel corso dell'algoritmo vengono anche assegnati ai nodi due valori interi: il tempo di scoperta $d$ e il tempo di
fine visita $f$, che permettono di determinare, rispettivamente, il momento in cui i nodi vengono scoperti e colorati
di grigio e il tempo in cui la visita di un nodo termina, colorandosi di nero.
Un attributo aggiuntivo, il predecessore $\pi$, permette di memorizzare il nodo da cui si \`e scoperto il nodo
corrente e, con esso, di ricostruire il cammino di visita sotto forma di albero, detto albero di visita in profondit\`a.
\newline

In figura~\ref{fig:dfs-example} \`e rappresentato un esempio di esecuzione della procedura DFS-VISIT su un grafo
diretto, eseguita a partire dal nodo con etichetta $9$, in cui lo stato del grafo \`e rappresentato ad ogni
incremento della variabile $time$. \newline

La procedura si mantiene in tempo lineare rispetto al numero di nodi e di archi del grafo, in quanto
ogni nodo viene visitato una sola volta e ogni arco viene esaminato al massimo una volta.

\paragraph{Ordinamento topologico}
Dato un grafo diretto aciclico $G = (V, E)$, un ordinamento topologico di $G$ \`e una particolare sequenza
dei suoi nodi $\langle v_1, v_2, \ldots, v_n \rangle$ tale che per ogni arco $(v_i, v_j) \in E$, $i < j$.
Si noti, quindi, che un ordinamento topologico di un grafo diretto pu\`o esistere solo se il grafo non contiene
cicli.
Un ordinamento topologico fornisce una disposizione tale che i nodi raggiungibili da un certo nodo $v_i$ vengano
disposti dopo di esso, e che i nodi che $v_j$ pu\`o raggiungere vengano disposti dopo di esso.

Una procedura algoritmica per il calcolo di un ordinamento topologico di $G$ pu\`o essere effettuata tramite una
visita in profondit\`a del grafo, raccogliendo in una lista concatenata i nodi in ordine decrescente di tempo
di fine visita, man mano che vengono visitati.
Per via del normale costo di una visita in profondit\`a, la complessit\`a di tale procedura \`e lineare al
numero di nodi e di archi del grafo \newline

\begin{algorithm}[H] \floatname{algorithm}{Algoritmo}
    \caption{KOSARAJU-ALGORITHM($G$)}\label{alg:cap2}
    \begin{algorithmic}[1]
        \State Esegui $DFS(G)$ per calcolare il tempo di fine visita $u.f$ per ogni nodo $u$ in $G$
        \State Calcola $G^T$
        \State Esegui $DFS(G^T)$, considerando $G.V$ in ordine decrescente di tempo di fine visita
        \State Fornisci in output ogni albero di visita in profondit\`a come componente fortemente connessa di $G$
    \end{algorithmic}
\end{algorithm}

Come descritto dallo pseudocodice, l'algoritmo di Kosaraju esegue due visite in profondit\`a, una sul grafo
originale $G$ e una sul grafo trasposto $G^T$, in cui la seconda visita viene effettuata secondo l'ordinamento
ottenuto dalla prima.
Le motivazioni per cui la procedura di visita a riga 3 permette di visitare una componente fortemente
connessa alla volta per ogni iterazione del ciclo for principale possono essere riassunte dai seguenti punti:
\begin{itemize}
    \item La condensazione di $G$ \`e un grafo aciclico, e quindi ammette un ordinamento topologico.
        Questo vuol dire che se esiste un cammino $u \rightsquigarrow v$ in $G$, con $u$ e $v$ nodi di componenti
        fortemente connesse distinte, allora di certo non pu\`o esistere un cammino $v \rightsquigarrow u$ in $G$,
        altrimenti i nodi in entrambe le componenti connesse sarebbero tutti mutualmente raggiungibili tra loro, e
        costituirebbero una unica componente fortemente connessa.
    \item Se $C$ e $C'$ sono componenti fortemente connesse di $G$ distinte ed esiste in $G$ un cammino
        $u \rightsquigarrow v$ con $u \in C$ e $v \in C'$, allora il tempo di fine visita massimo tra i nodi di $C$
        sar\`a maggiore del tempo di fine visita massimo tra i nodi di $C'$.
        Infatti, supponendo che venga visitato prima $u$, tutti i nodi raggiungibili da $u$, incluso $v$ e tutti i nodi
        in $C'$, saranno visitati prima che la visita di $u$ termini.
        Al contempo, supponendo che venga visitato prima $v$, allora nessun nodo di $C$ verr\`a visitato prima che
        la visita di $v$ termini, in quanto, come detto nel primo punto, l'esistenza di un cammino
        $u \rightsquigarrow v$ esculde l'esistenza di un cammino $v \rightsquigarrow u$.
    \item Considerare i nodi in ordine decrescente di tempo di fine visita permette di visitare prima le componenti
        fortemente connesse meno \"profonde\", ovvero quelle che non contengono nodi raggiungibili da altre
        componenti fortemente connesse.
        In altre parole, le componenti meno profonde sono le prime ad apparire nell'ordinamento topologico della
        condensazione.
    \item Invertendo il senso degli archi e visitando prima le componenti fortemente connesse meno profonde in $G$, si
        garantisce che non si possano raggiungere nodi di altre componenti, ma solo nodi della stessa componente.
        Invertendo il senso degli archi, infatti, si inverte l'ordinamento topologico della condensazione, visitando
        prima le compoenenti che si trovano in fondo al nuovo ordinamento.
\end{itemize}

\begin{figure}
    \resizebox{!}{4cm}{
        \begin{tikzpicture}
    %comp 1
    \node[mynode](n1) at (0,0){1};
    \node[mynode](n2) at (1.2,-1){2};
    \node[mynode](n3) at (1.2, 1){4};
    \node[mynode](n4) at (2.4, 0){3};
    \draw[myarrow](n1)--(n2);
    \draw[myarrow](n3)--(n1);
    \draw[myarrow](n4)--(n3);
    \draw[myarrow](n2)--(n4);
    %comp 2
    \node[mynode](n5) at (4,1.2){5};
    \node[mynode](n6) at (4,0){6};
    \node[mynode](n7) at (4, -1.2){7};
    \draw[myarrow](n6) -- (n5);
    \draw[myarrow](n5) to[out=3,in=3] (n7);
    \draw[myarrow](n7) -- (n6);

    %comp edges
    \draw[myarrow](n3) -- (n5);
    \draw[myarrow](n2) -- (n7);
    %comp 3
    \node[mynode](n8) at (6, 2){8};
    %comp 4
    \node[mynode](n9) at (8, 2){9};
    \node[mynode](na) at (8.5, 0){10};
    \draw[myarrow](n9) -- (n8);
    \draw[myarrow](na) to[out=3,in=3] (n9);
    \draw[myarrow](n9) -- (na);
    \draw[myarrow](n8) -- (n5);

    \node[below=10mm] at (5, -1.2) {(a)};
\end{tikzpicture} \quad \quad \quad
\begin{tikzpicture}
    %comp 1
    \node[mynode, label=above:{$13/20$}, fill=black](n1) at (0,0){\textcolor{white}{1}};
    \node[mynode, label=above:{$14/19$}, fill=black](n2) at (1.2,-1){\textcolor{white}{2}};
    \node[mynode, label=above:{$16/17$}, fill=black](n3) at (1.2, 1){\textcolor{white}{4}};
    \node[mynode, label=above:{$15/18$}, fill=black](n4) at (2.4, 0){\textcolor{white}{3}};
    \draw[myarrow](n1)--(n2);
    \draw[myarrow](n3)--(n1);
    \draw[myarrow](n4)--(n3);
    \draw[myarrow](n2)--(n4);
    %comp 2
    \node[mynode, label=above:{$5/10$}, fill=black](n5) at (4,1.2){\textcolor{white}{5}};
    \node[mynode, label=right:{$7/8$}, fill=black](n6) at (4,0){\textcolor{white}{6}};
    \node[mynode, label=below:{$6/9$}, fill=black](n7) at (4, -1.2){\textcolor{white}{7}};
    \draw[myarrow](n6) -- (n5);
    \draw[myarrow](n5) to[out=3,in=3] (n7);
    \draw[myarrow](n7) -- (n6);

    %comp edges
    \draw[myarrow](n3) -- (n5);
    \draw[myarrow](n2) -- (n7);
    %comp 3
    \node[mynode, label=above:{$4/11$}, fill=black](n8) at (6, 2){\textcolor{white}{8}};
    %comp 4
    \node[mynode, label=above:{$1/12$}, fill=black](n9) at (8, 2){\textcolor{white}{9}};
    \node[mynode, label=below:{$2/3$}, fill=black](na) at (8.5, 0){\textcolor{white}{10}};
    \draw[myarrow](n9) -- (n8);
    \draw[myarrow](na) to[out=3,in=3] (n9);
    \draw[myarrow](n9) -- (na);
    \draw[myarrow](n8) -- (n5);

    % label for the second graph
    \node[align=center, xshift=2cm, yshift=2.8cm] (label2) {$\langle 1, 2, 3, 4, 9, 8, 5, 7, 6, 10 \rangle$};
    \node[below=10mm] at (5, -1.2) {(b)};
\end{tikzpicture}}
    \resizebox{!}{4cm}{
        \begin{tikzpicture}
    %comp 1
    \node[mynode](n1) at (0,0){1};
    \node[mynode](n2) at (1.2,-1){2};
    \node[mynode](n4) at (1.2, 1){4};
    \node[mynode](n3) at (2.4, 0){3};
    \draw[myarrow](n2)--(n1);
    \draw[myarrow](n1)--(n4);
    \draw[myarrow](n4)--(n3);
    \draw[myarrow](n3)--(n2);
    %comp 2
    \node[mynode](n5) at (4,1.2){5};
    \node[mynode](n6) at (4,0){6};
    \node[mynode](n7) at (4, -1.2){7};
    \draw[myarrow](n5) -- (n6);
    \draw[myarrow](n7) to[out=3,in=3] (n5);
    \draw[myarrow](n6) -- (n7);

    %comp edges
    \draw[myarrow](n5) -- (n4);
    \draw[myarrow](n7) -- (n2);
    %comp 3
    \node[mynode](n8) at (6, 2){8};
    %comp 4
    \node[mynode](n9) at (8, 2){9};
    \node[mynode](na) at (8.5, 0){10};
    \draw[myarrow](n8) -- (n9);
    \draw[myarrow](n9) to[out=3,in=3] (na);
    \draw[myarrow](na) -- (n9);
    \draw[myarrow](n5) -- (n8);

    \node[align=left, xshift=2cm, yshift=2.8cm] (label2) {$\langle 1, 2, 3, 4, 9, 8, 5, 7, 6, 10 \rangle$};
    \node[below=10mm] at (5, -1.2) {(c)};
\end{tikzpicture} \quad \quad \quad
\begin{tikzpicture}
    %comp 1
    \node[mynode, fill={rgb:red,4;green,2;yellow,1}](n1) at (0,0){\textcolor{white}{1}};
    \node[mynode, fill={rgb:red,4;green,2;yellow,1}](n2) at (1.2,-1){\textcolor{white}{2}};
    \node[mynode, fill={rgb:red,4;green,2;yellow,1}](n4) at (1.2, 1){\textcolor{white}{4}};
    \node[mynode, fill={rgb:red,4;green,2;yellow,1}](n3) at (2.4, 0){\textcolor{white}{3}};
    \draw[myarrow, color={rgb:red,4;green,2;yellow,1}](n2)--(n1);
    \draw[myarrow, color={rgb:red,4;green,2;yellow,1}](n1)--(n4);
    \draw[myarrow, color={rgb:red,4;green,2;yellow,1}](n4)--(n3);
    \draw[myarrow, color={rgb:red,4;green,2;yellow,1}](n3)--(n2);
    %comp 2
    \node[mynode, fill=red](n5) at (4,1.2){\textcolor{white}{5}};
    \node[mynode, fill=red](n6) at (4,0){\textcolor{white}{6}};
    \node[mynode, fill=red](n7) at (4, -1.2){\textcolor{white}{7}};
    \draw[myarrow, red](n5) -- (n6);
    \draw[myarrow, red](n7) to[out=3,in=3] (n5);
    \draw[myarrow, red](n6) -- (n7);
    %comp edges
    \draw[myarrow, color={rgb:red,1;blue,2}](n5) -- (n4);
    \draw[myarrow, color={rgb:red,1;blue,2}](n7) -- (n2);
    %comp 3
    \node[mynode, fill=green!50!red](n8) at (6, 2){\textcolor{white}{8}};
    %comp 4
    \node[mynode, fill={rgb:red,1;green,2;blue,5}](n9) at (8, 2){\textcolor{white}{9}};
    \node[mynode, fill={rgb:red,1;green,2;blue,5}](na) at (8.5, 0){\textcolor{white}{10}};
    \draw[myarrow, color={rgb:red,4;green,1;blue,3}](n8) -- (n9);
    \draw[myarrow, color={rgb:red,1;green,2;blue,5}](n9) to[out=3,in=3] (na);
    \draw[myarrow, color={rgb:red,1;green,2;blue,5}](na) -- (n9);
    \draw[myarrow, color={rgb:red,2;green,1}](n5) -- (n8);

    \node[align=left, xshift=0.3cm, yshift=2.8cm] (label2) {$\langle \rangle$};

    \node[below=10mm] at (5, -1.2) {(d)};
\end{tikzpicture}}
    \caption{Esempio di esecuzione dell'algoritmo di Kosaraju su un grafo diretto}
    \label{fig:kozaraju_example}
\end{figure}

In figura~\ref{fig:kozaraju_example} sono rappresentate le fasi rilevanti dell'algoritmo di Kozaraju
applicato al grafo in (a).
In (b) \`e rappresentato lo stato del grafo al termine della procedura DFS, assieme alla lista dei nodi ordinata
per tempi di fine visita.
In (c) il grafo trasposto ottenuto invertendo gli archi del grafo in (a).
In (d) lo stato del grafo trasposto in (c) al termine della seconda procedura di visita in
profondit\`a, dove colori dello stesso colore rappresentano nodi appartenenti alla stessa componente fortemente
connessa fornita in output.

\paragraph{Complessit\`a dell'algoritmo}
L'algoritmo di Kosaraju ha una complessit\`a temporale di $\Theta(V + E)$, in quanto:
\begin{itemize}
    \item A riga 1 viene eseguita una ricerca in profondit\`a tradizionale, quindi con costo $\Theta(|V| + |E|)$.
    \item A riga 2 viene costruito il grafo trasposto $G^T$ e come gi\`a detto, la sua costruzione pu\`o avvenire in
    un tempo $\Theta(|V| + |E|)$, iterando prima sull'insieme dei nodi di $G$ e poi sui suoi archi, invertendone
    l'orientamento.
    \item L'ordinamento dei nodi di $G^T$ pu\`o essere realizzato nel corso della visita a riga 1 attraverso la
    costruizione di una lista concatenata, per cui i singoli nodi possono essere aggiunti in testa alla lista nel
    momento in cui vengono colorati di nero, in tempo costante, per un totale di $\Theta(|V|)$, e non ci sono costi
    aggiuntivi per la costruzione dell'ordinamento.
    \item A riga 3 viene eseguita la procedura DFS che corrisponde ad una visita in profondit\`a di $G^T$, quindi
    sempre di complessit\`a $\Theta(|V| + |E|)$.
\end{itemize}
