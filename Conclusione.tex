\chapter*{Conclusioni e Sviluppi Futuri}\addcontentsline{toc}{chapter}{Conclusioni e Sviluppi Futuri}

È stata poi eseguita l'identificazione delle entità nominate (NER) per individuare e classificare le entità nominate
all'interno del testo. Questo processo aiuta a riconoscere e categorizzare elementi come persone, luoghi,
organizzazioni e altre entità specifiche del dominio rilevanti per il contenuto dei sogni. Il NER può rivelare
temi importanti o elementi ricorrenti nei sogni di Emma, come luoghi frequenti o figure significative.
È stata inoltre condotta un'analisi della frequenza delle parole come parte dell'analisi preliminare.
Questo passaggio implica il conteggio delle occorrenze di ogni parola unica nel corpus, fornendo intuizioni sui
termini più comuni utilizzati nei sogni di Emma. Tale analisi può mettere in evidenza temi predominanti, emozioni o
oggetti che appaiono frequentemente nelle narrazioni dei sogni, aiutando a comprendere la struttura linguistica
di base delle narrazioni dei sogni.