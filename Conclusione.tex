\chapter*{Conclusioni e Sviluppi Futuri}\addcontentsline{toc}{chapter}{Conclusioni e Sviluppi Futuri}

Come appare evidente dai risultati di queste tecniche di analisi basate su grafi multi-livello, la loro applicazione
potrebbe rivelare strutture celate all'interno delle narrazioni dei sogni, che siano di natura sintattica, come
l'organizzazione del testo a livelli di astrazione superiore, o semantica, come la presenza di temi e schemi di
pensiero ricorrenti o cluster di concetti correlati.
Nell'ambito dell'analisi dei sogni, gli sviluppi futuri auspicati potrebbero riguardare l'applicazione di tali
tecniche di analisi a dati categorizzati, per rivelare correlazioni tra le strutture del grafo multi-livello
e gli stati psicologici dei sognatori, oltre che la valutazione di miglioramenti alle tecniche di contrazione e
analisi che siano ad-hoc per questo specifico dominio, come potrebbe essere la considerazione del peso degli archi
nelle funzioni di contrazione.

Tuttavia, come evidenziato da questi risultati preliminari, in generale l'approccio analitico multi-livello offre
una prospettiva globale e dinamica che va oltre l'analisi a livello singolo, rivelando strutture e relazioni complesse
all'interno dei dati.
Per questo motivo, sarebbe altrettanto interessante esplorare il comportamento di questa struttura dati in altri
domini applicativi.

Infine, un ulteriore sviluppo potrebbe riguardare la progettazione di un sistema di aggiornamento dinamico
della struttura dati che permetta di aggiungere o rimuovere nuovi nodi o archi alla base della struttura
gerarchica in modo da evitare di dover ricalcolare l'intera struttura ad ogni modifica.
Questa funzionalità potrebbe essere utile per un'ulteriore tipologia di analisi che riguardi la
sensitività dei livelli più astratti alle modifiche dei dati di partenza che, declinata nell'ambito dell'analisi
dei sogni, potrebbe rivelare l'influenza di singole parole o collegamenti nella struttura complessiva.
