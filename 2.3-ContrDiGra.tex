\section{Contrazione di Grafi}\label{sec:contrazione-di-grafi}

Nella teoria dei grafi la contrazione di un grafo \`e un'operazione che permette di ridurre la dimensione di un grafo
senza alterarne la struttura fondamentale. \newline
La contrazione di archi o di sottoinsiemi di nodi \`e un operazione fondamentale nella teoria dei grafi minori, dove
si studiano le propriet\`a di un grafo in relazione alla presenza di sottostrutture minori ottenibili attraverso
rimozione di archi e nodi o contrazioni. \newline
Queste tecniche di contrazione trovano applicazione in tutti quei casi in cui si vuole semplificare un grafo
identificando i vertici che possono essere considerati equivalenti in relazione ad una certa propriet\`a,
e risultano essere utili in svariati problemi di ottimizzazione e partizionamento di grafi.
In letteratura, le operazioni di contrazione di grafi sono state utilizzate anche a scopo di compressione di grafi,
al fine di renderli pi\`u compatti e trattabili con algoritmi di analisi altrimenti troppo costosi,
individuando schemi di contrazione d'interesse e cercando di evitare perdita di
informazione~\cite{10.1145/3448016.3452797}. \newline


\subsection{Contrazione di archi}\label{subsec:contrazione-di-archi}

La contrazione di archi, spesso riferita come \textbf{contrazione di spigoli}, di un grafo orientato $G = (V, E)$ \`e
un'operazione che consiste nella rimozione di un  arco $e = (u, v) \in E$ e nella simultanea fusione dei nodi $u$ e
$v$ in un unico nodo $w$.
Quando ci\`o avviene, tutti gli archi che entrano in $u$ e $v$ diventano archi entranti in $w$, e, analogamente,
tutti gli archi che escono da $u$ e $v$ diventano archi uscenti da $w$.
Il risultato di una tale operazione \`e, quindi, un nuovo grafo ottenuto da $G$ mediante la contrazione
dell'arco $e$, che pu\`o essere indicato con $G/e$ (da non confondersi con la sottrazione insiemistica $\setminus$).
Si noti che, secondo la definzione data, una tale operazione applicata ad un grafo orientato semplice pu\`o risultare
in un grafo con archi multipli e cappi, a seconda della struttura del grafo iniziale, e per questo \`e
spesso previsto nella definzione di contrazione di archi che vengano applicate le ulteriori operazioni necessarie ad
ottenere come risultato un nuovo grafo semplice. \newline

\begin{definition}[Contrazione di archi]
Sia $G = (V, E)$ un grafo orientato e sia $e = (u, v) \in E$ un arco di $G$ con $u \neq v$,
sia $f$ una funzione su $V$ che associa ogni nodo in $V \setminus \{u, v\}$ a se stesso, o ad un nuovo nodo $w$
altrimenti. \newline
La contrazione di $e$ su $G$ \`e un nuovo grafo $G' = (V', E')$ dove:
\begin{itemize}
    \item $V' = (V \setminus \{u, v\}) \cup \{w\}$ con $w \notin V$
    \item $E' = \{(f(x), f(y)) \mid (x, y) \in E \setminus \{e\}\}$
\end{itemize}
\end{definition}

In figura~\ref{fig:edge-contraction-example} \`e mostrato un esempio di contrazione di un arco $(u, v)$ in un nuovo
nodo $w$ in un grafo orientato, che include la rimozione di archi multipli e di cappi.
Pi\`u in generale, una tale operazione pu\`o essere eeguita su un insieme di archi, contraendo ciacuno di essi in
un qualsiasi ordine.

\begin{figure}[h]
    \centering
    \begin{tikzpicture}
  % Styles for the nodes and edges
  \tikzstyle{vertex} = [draw, circle, inner sep=2pt, minimum size=12pt]
  \tikzstyle{edge} = [->, >={Stealth[round]}, thick]

  % Left Graph
  \node[vertex, label=left:u] (u) at (0, 1) {};
  \node[] (u2) at (-0.2, 0.8) {};
  \node[vertex, label=left:v] (v) at (0, -1) {};
  \node[] (v2) at (-0.2, -0.8) {};

  \node[vertex] (a) at (-1.5, 0.5) {};
  \node[vertex] (b) at (1.5, 0.5) {};
  \node[vertex] (c) at (-0.5, 2.2) {};
  \node[vertex] (d) at (1, 1.5) {};
  \node[vertex] (e) at (-1, -2) {};
  \node[vertex] (f) at (1.5, -0.7) {};
  \node[vertex] (g) at (0.5, -2.2) {};

  \draw[edge] (u) -- (a);
  \draw[edge] (u) -- (b);
  \draw[edge] (c) -- (u);
  \draw[edge] (u) -- (d);
  \draw[edge] (f) -- (u);
  \draw[edge] (c) -- (a);

  \draw[edge] (v) -- (a);
  \draw[edge] (e) -- (v);
  \draw[edge] (v) -- (f);
  \draw[edge] (v) -- (g);
  \draw[edge] (g) -- (e);

  \draw[edge] (u) -- (v);
  \draw[->, red] (u2) -- ++( 0, -0.5);
  \draw[->, red] (v2) -- ++( 0, 0.5);

  % Right Graph
  \node[vertex, label=left:w] (w) at (6, 0) {};

  \node[vertex] (a2) at (4.5, 0.5) {};
  \node[vertex] (b2) at (7.5, 0.5) {};
  \node[vertex] (c2) at (5.5, 2.2) {};
  \node[vertex] (d2) at (7, 1.5) {};
  \node[vertex] (e2) at (5, -2) {};
  \node[vertex] (f2) at (7.5, -0.7) {};
  \node[vertex] (g2) at (6.5, -2.2) {};

  \draw[edge] (w) -- (a2);
  \draw[edge] (w) -- (b2);
  \draw[edge] (c2) -- (w);
  \draw[edge] (w) -- (d2);
  \draw[edge] (f2) -- (w);
  \draw[edge] (c2) -- (a2);

  \draw[edge] (w) -- (a2);
  \draw[edge] (e2) -- (w);
  \draw[edge] (w) -- (f2);
  \draw[edge] (w) -- (g2);
  \draw[edge] (g2) -- (e2);

\end{tikzpicture}
    \caption{Esempio di contrazione di un arco in un grafo orientato}
    \label{fig:edge-contraction-example}
\end{figure}

\subsection{Contrazione di sottografi}\label{subsec:contrazione-di-sottografi}
Un'operazione simile alla contrazione di archi, ma pi\`u generale, \`e la \textbf{contrazione di vertici}
(o \textbf{identificazione di vertici}) di un grafo.
Essa pu\`o essere vista come una generalizzazione della contrazione di archi, in quanto rimuove la restrizione che
la coppia di nodi da contrarre sia adiacente, rendendo la contrazione per archi un suo caso particolare.
Si immagini, pertanto, di avere il grafo a sinistra della figura~\ref{fig:subgraph-contraction-example} privato,
per\`o, dell'arco $(u, v)$.
La contrazione per vertici permetterebbe di contrarre la coppia non adiacente di nodi $u$ e $v$, risultando,
comunque, nel grafo a destra della figura~\ref{fig:subgraph-contraction-example}. \newline

L' operazione di contrazione di vertici pu\`o essere generalizzata nella \textbf{contrazione di sottografi},
un'operazione che permette di contrarre un qualsiasi sottoinsieme di nodi di un grafo in un unico nodo.
Dato un grafo $G = (E_G, V_G)$ ed un suo sottografo $H = (V_H, E_H)$, quindi, il grafo risultante dalla contrazione
di $H$ mantiene tutti gli archi incidenti su coppie di nodi in $E_G \setminus E_H$, sostituendo
quegli archi incidenti tra nodi in $V_G \setminus V_H$ e $V_H$ con nuovi archi incidenti sul nuovo nodo contratto.

\begin{definition}[Contrazione di sottografi]
    Sia $G = (V, E)$ un grafo orientato, sia $W \subseteq V$ un sottoinsieme di nodi di $G$, sia $H = G[W] = (W, F)$
    il sottografo indotto da $W$ in $G$.
    Sia $f$ una funzione su $V$ che associa ogni nodo in $V \setminus W$ a se stesso, o ad un nuovo nodo $w$
    altrimenti.
    La contrazione di $H$ su $G$ \`e un nuovo grafo $G' = (V', E')$ dove:
    \begin{itemize}
        \item $V' = (V \setminus W) \cup \{w\}$ con $w \notin V$
        \item $E' = \{(f(u), f(v)) \mid (u, v) \in E \setminus F\}$
    \end{itemize}
\end{definition}

In figura~\ref{fig:subgraph-contraction-example} \`e mostrato un esempio di contrazione di un sottografo
$G[\{v_1, v_2, v_3, v_4\}]$ del grafo orientato $G$ in un nuovo nodo $w$, che include la rimozione di archi multipli
e di cappi. \newline

\begin{figure}[h]
    \centering
    \begin{tikzpicture}[scale=1.5]

% Define the styles for the nodes and edges
\tikzstyle{vertex} = [draw, circle, inner sep=2pt, minimum size=18pt]
\tikzstyle{edge} = [->, >={Stealth[round]}, thick]
\tikzset{thick vertex/.style = {draw, circle, inner sep=2pt, minimum size=18pt, line width=1.7pt}}]
\tikzset{thick edge/.style = {->, >={Stealth[round]}, line width=1.7pt}}

% Left graph
\node[] (g) at (3.2,1.8) {$G$};

\node[thick vertex] (v1) at (0,0) {$v_1$};
\node[thick vertex] (v2) at (0,2) {$v_2$};
\node[thick vertex] (v3) at (0.75,1) {$v_3$};
\node[thick vertex] (v4) at (1.5,0) {$v_4$};
\node[vertex] (v5) at (1.5,2) {$v_5$};
\node[vertex] (v6) at (2.5,0) {$v_6$};
\node[vertex] (v7) at (2.5,2) {$v_7$};
\node[vertex] (v8) at (3.2,1) {$v_8$};

% Draw edges for left graph
\draw[thick edge] (v1) -- (v2);
\draw[thick edge] (v1) -- (v3);
\draw[thick edge] (v2) -- (v3);
\draw[thick edge] (v3) -- (v4);
\draw[thick edge] (v4) -- (v1);
\draw[edge] (v2) -- (v5);
\draw[edge] (v3) -- (v5);
\draw[edge] (v4) -- (v5);
\draw[edge] (v4) -- (v6);
\draw[edge] (v5) -- (v7);
\draw[edge] (v6) -- (v7);
\draw[edge] (v6) -- (v8);
\draw[edge] (v7) -- (v4);
\draw[edge] (v7) -- (v8);
\draw[edge] (v6) -- (v8);

% Right graph
\node[] (g2) at (8.2,1.8) {$G'$};

\node[thick vertex] (w) at (5.8,0.7) {$w$};
\node[vertex] (v5) at (6.5,2) {$v_5$};
\node[vertex] (v6) at (7.5,0) {$v_6$};
\node[vertex] (v7) at (7.5,2) {$v_7$};
\node[vertex] (v8) at (8.2,1) {$v_8$};

% Draw edges for right graph
\draw[edge] (w) -- (v5);
\draw[edge] (w) -- (v6);
\draw[edge] (v5) -- (v7);
\draw[edge] (v6) -- (v7);
\draw[edge] (v6) -- (v8);
\draw[edge] (v7) -- (w);
\draw[edge] (v7) -- (v8);
\draw[edge] (v6) -- (v8);

% Draw the arrow
\draw[-{Stealth[length=3mm, width=2mm]}, thick] (4,1.1) -- (5,1.1);

\end{tikzpicture}
    \caption{Esempio di contrazione di un sottografo in un grafo orientato}
    \label{fig:subgraph-contraction-example}
\end{figure}

Alla luce delle definizioni delle operazioni presentate, valgono le seguenti considerazioni:
\begin{itemize}
    \item Il risultato della contrazione di una coppia di nodi adiacienti su un certo grafo $G$ pu\`o produrre un
    grafo isomorfo a quello della contrazione di una coppia di nodi non adiacenti in un altro grafo $G'$ non isomorfo
    a $G$.
    E' il caso precedentemente considerato applicato al grafo a sinistra in figura~\ref{fig:edge-contraction-example}.
    \item Il risultato della contrazione di un sottografo su un certo grafo $G$ pu\`o produrre un grafo isomorfo
    a quello della contrazione di un sottografo in un altro grafo $G'$ non isomorfo a $G$.
    Come esempio analogo, basta considerare un grafo $J$ ottenuto a apartire dal grafo $G$ a sinistra in
    figura~\ref{fig:subgraph-contraction-example} rimuovendo il nodo $v_1$ e i suoi archi incidenti.
    I grafi risultanti dalla contrazione di $G[\{v_1, v_2, v_3, v_4\}]$ in $G$ e dalla contrazione di
    $J[\{v_2, v_3, v_4\}]$ in $G'$ sono certamente isomorfi.
\end{itemize}

Questo significa che la contrazione di vertici e di sottografi non sono operazioni invertibili, in quanto
rappresentano funzioni suriettive, e quindi non iniettive.
Di fatti queste operazioni non mantengono alcuna informazione legata alla struttura originale del grafo su cui
sono applicate.
Come mostrato nei prossimi capitoli, tra gli obiettivi della definizione del grafo multi-livello, vi \`e proprio
quello di mantenere le informazioni legate alla struttura dei grafi a cui sono applicate contrazioni, permettendo
anche operazioni di decontrazione.

\subsection{Grafi quoziente}\label{subsec:grafi-quoziente}

Nella teoria dei grafi, un grafo quoziente \`e una visione astratta di un grafo partizionato in sottoinsiemi di nodi
che rappresenta le relazioni tra tali sottoinsiemi.
In un grafo quoziente $G'$ ottenuto a partire da un grafo $G = (V, E)$, i nodi rappresentano \"blocchi\" di nodi di $G$
che fanno parte dello stesso insieme per una qualche partizione di $V$.
Per quanto riguarda gli archi di $G'$, dati due blocchi di nodi $B_1$ e $B_2$ in $G'$, un arco tra $B_1$ e $B_2$ sta
ad indicare la presenza di almeno un arco tra un nodo di $B_1$ e un nodo di $B_2$ in $G$. \newline

Se intuitivamente si potrebbe dire che il grafo quoziente permette di accorpare gruppi di nodi e archi tra loro
per formare un nuovo grafo, una descrizione pi\`u formale utilizzerebbe il concetto di contrazione di sottografi,
definendo il grafo quoziente come il risultato delle contrazioni dei sottografi indotti dalla data partizione di nodi.

\begin{definition}[Grafo Quoziente]
Sia $G = (V, E)$ un grafo orientato, sia $P \subseteq \mathcal{P}(V)$ una partizione di $V$, sia $R$ la relazione
d'equivalenza su $V$ indotta dalla partizione $P$.
Il grafo quoziente di $G$ rispetto a $P$ \`e il grafo $G' = (V', E')$ dove:
    \begin{itemize}
        \item $V'$ \`e l'insieme quoziente $V/R$, ovvero l'insieme delle classi di equivalenza di $R$ su $V$.
        \item $E' = \{([u]_R, [v]_R) \mid (u, v) \in E\}$, dove $[u]_R$ e $[v]_R$ sono rispettivamente le classi di
        equivalenza dei nodi $u$ e $v$ rispetto a $R$.
    \end{itemize}
\end{definition}

La figura~\ref{fig:quotient-graph-example} mostra un esempio di grafo quoziente sulla destra ottenuto a partire dal
grafo orientato sulla sinistra e una partizione $P = \{A, B, C\}$. \newline

Come evidente dalla definzione, il nome del grafo quoziente \`e dovuto al fatto che la sua struttura \`e
strettamente legata all'insieme quoziente di una qualche relazione di equivalenza definita sui nodi del grafo.
Sebbene, assieme al grafo di partenza, l'ingrediente fondamentale per la definizione di un grafo quoziente sia la
una partizione dei suoi nodi, una relazione di equivalenza sugli stessi sarebbe un parametro equivalente, in quanto
ogni relazione di equivalenza induce una partizione degli elementi del suo dominio in classi di equivalenza. \newline

\begin{figure}[h]
    \centering
    \begin{tikzpicture}[x=0.75pt,y=0.75pt,yscale=-1,xscale=1,graph node/.style={circle, draw, inner sep=2pt}, >={Stealth}]
%uncomment if require: \path (0,193); %set diagram left start at 0, and has height of 193

%Shape: Ellipse [id:dp5716078870574002]
\draw   (157.72,126.37) .. controls (157.72,122.18) and (161.11,118.78) .. (165.3,118.78) .. controls (169.49,118.78) and (172.89,122.18) .. (172.89,126.37) .. controls (172.89,130.56) and (169.49,133.96) .. (165.3,133.96) .. controls (161.11,133.96) and (157.72,130.56) .. (157.72,126.37) -- cycle ;
%Straight Lines [id:da030036293703488592]
\draw    (162.59,132.58) -- (119.64,155.58) ;
\draw [shift={(117,157)}, rotate = 331.82] [fill={rgb, 255:red, 0; green, 0; blue, 0 }  ][line width=0.08]  [draw opacity=0] (7.14,-3.43) -- (0,0) -- (7.14,3.43) -- cycle    ;
%Shape: Ellipse [id:dp4694008742489806]
\draw   (96.69,48.05) .. controls (96.69,43.86) and (100.09,40.46) .. (104.28,40.46) .. controls (108.47,40.46) and (111.87,43.86) .. (111.87,48.05) .. controls (111.87,52.24) and (108.47,55.64) .. (104.28,55.64) .. controls (100.09,55.64) and (96.69,52.24) .. (96.69,48.05) -- cycle ;
%Shape: Ellipse [id:dp43075157044098433]
\draw   (157.8,159.08) .. controls (157.8,154.89) and (161.2,151.49) .. (165.39,151.49) .. controls (169.58,151.49) and (172.98,154.89) .. (172.98,159.08) .. controls (172.98,163.27) and (169.58,166.67) .. (165.39,166.67) .. controls (161.2,166.67) and (157.8,163.27) .. (157.8,159.08) -- cycle ;
%Straight Lines [id:da7735175864839863]
\draw    (157.8,159.08) -- (122.91,161.4) ;
\draw [shift={(119.91,161.6)}, rotate = 356.19] [fill={rgb, 255:red, 0; green, 0; blue, 0 }  ][line width=0.08]  [draw opacity=0] (7.14,-3.43) -- (0,0) -- (7.14,3.43) -- cycle    ;
%Shape: Ellipse [id:dp8331515179192863]
\draw   (103.74,126.6) .. controls (103.74,122.41) and (107.14,119.01) .. (111.33,119.01) .. controls (115.52,119.01) and (118.91,122.41) .. (118.91,126.6) .. controls (118.91,130.79) and (115.52,134.19) .. (111.33,134.19) .. controls (107.14,134.19) and (103.74,130.79) .. (103.74,126.6) -- cycle ;
%Straight Lines [id:da8805781538281761]
\draw    (165.3,133.96) -- (165.38,148.49) ;
\draw [shift={(165.39,151.49)}, rotate = 269.72] [fill={rgb, 255:red, 0; green, 0; blue, 0 }  ][line width=0.08]  [draw opacity=0] (7.14,-3.43) -- (0,0) -- (7.14,3.43) -- cycle    ;
%Shape: Ellipse [id:dp11540879760206746]
\draw   (24.41,89.59) .. controls (24.41,85.4) and (27.81,82) .. (32,82) .. controls (36.19,82) and (39.59,85.4) .. (39.59,89.59) .. controls (39.59,93.78) and (36.19,97.17) .. (32,97.17) .. controls (27.81,97.17) and (24.41,93.78) .. (24.41,89.59) -- cycle ;
%Straight Lines [id:da7303103534852602]
\draw    (90.93,73.6) -- (90.93,73.6) -- (59.6,55.5) ;
\draw [shift={(57,54)}, rotate = 30.02] [fill={rgb, 255:red, 0; green, 0; blue, 0 }  ][line width=0.08]  [draw opacity=0] (7.14,-3.43) -- (0,0) -- (7.14,3.43) -- cycle    ;
%Shape: Ellipse [id:dp18097418973582324]
\draw   (42.46,49.02) .. controls (42.46,44.83) and (45.85,41.43) .. (50.04,41.43) .. controls (54.23,41.43) and (57.63,44.83) .. (57.63,49.02) .. controls (57.63,53.21) and (54.23,56.6) .. (50.04,56.6) .. controls (45.85,56.6) and (42.46,53.21) .. (42.46,49.02) -- cycle ;
%Straight Lines [id:da4155132765855869]
\draw    (57.63,49.02) -- (93.69,48.12) ;
\draw [shift={(96.69,48.05)}, rotate = 178.58] [fill={rgb, 255:red, 0; green, 0; blue, 0 }  ][line width=0.08]  [draw opacity=0] (7.14,-3.43) -- (0,0) -- (7.14,3.43) -- cycle    ;
%Straight Lines [id:da1262374391562433]
\draw    (32,82) -- (44.09,57.83) ;
\draw [shift={(45.43,55.14)}, rotate = 116.57] [fill={rgb, 255:red, 0; green, 0; blue, 0 }  ][line width=0.08]  [draw opacity=0] (7.14,-3.43) -- (0,0) -- (7.14,3.43) -- cycle    ;
%Straight Lines [id:da572291236631852]
\draw    (105.14,120.57) -- (78.69,87.9) -- (54.94,58.92) ;
\draw [shift={(53.04,56.6)}, rotate = 50.66] [fill={rgb, 255:red, 0; green, 0; blue, 0 }  ][line width=0.08]  [draw opacity=0] (7.14,-3.43) -- (0,0) -- (7.14,3.43) -- cycle    ;
%Curve Lines [id:da021365012814849704]
\draw    (98.95,42.72) .. controls (89.68,35.14) and (74.55,33.42) .. (61.85,39.48) .. controls (60.82,39.97) and (59.81,40.51) .. (58.82,41.1) ;
\draw [shift={(56.34,42.72)}, rotate = 324.46] [fill={rgb, 255:red, 0; green, 0; blue, 0 }  ][line width=0.08]  [draw opacity=0] (7.14,-3.43) -- (0,0) -- (7.14,3.43) -- cycle    ;
%Shape: Ellipse [id:dp776996888716095]
\draw   (90.41,78.59) .. controls (90.41,74.4) and (93.81,71) .. (98,71) .. controls (102.19,71) and (105.59,74.4) .. (105.59,78.59) .. controls (105.59,82.78) and (102.19,86.17) .. (98,86.17) .. controls (93.81,86.17) and (90.41,82.78) .. (90.41,78.59) -- cycle ;
%Shape: Ellipse [id:dp46756955428949243]
\draw   (104.74,161.6) .. controls (104.74,157.41) and (108.14,154.01) .. (112.33,154.01) .. controls (116.52,154.01) and (119.91,157.41) .. (119.91,161.6) .. controls (119.91,165.79) and (116.52,169.19) .. (112.33,169.19) .. controls (108.14,169.19) and (104.74,165.79) .. (104.74,161.6) -- cycle ;
%Straight Lines [id:da11987408246874143]
\draw    (112.33,154.01) -- (111.48,137.18) ;
\draw [shift={(111.33,134.19)}, rotate = 87.11] [fill={rgb, 255:red, 0; green, 0; blue, 0 }  ][line width=0.08]  [draw opacity=0] (7.14,-3.43) -- (0,0) -- (7.14,3.43) -- cycle    ;
%Straight Lines [id:da7165440164643189]
\draw    (118.91,126.6) -- (154.72,126.39) ;
\draw [shift={(157.72,126.37)}, rotate = 179.66] [fill={rgb, 255:red, 0; green, 0; blue, 0 }  ][line width=0.08]  [draw opacity=0] (7.14,-3.43) -- (0,0) -- (7.14,3.43) -- cycle    ;
%Straight Lines [id:da8702436048031414]
\draw    (106.33,156.19) -- (39.27,97.97) ;
\draw [shift={(37,96)}, rotate = 40.96] [fill={rgb, 255:red, 0; green, 0; blue, 0 }  ][line width=0.08]  [draw opacity=0] (7.14,-3.43) -- (0,0) -- (7.14,3.43) -- cycle    ;
%Straight Lines [id:da01851647212321983]
\draw    (111.33,119.01) -- (101.88,88.04) ;
\draw [shift={(101,85.17)}, rotate = 73.03] [fill={rgb, 255:red, 0; green, 0; blue, 0 }  ][line width=0.08]  [draw opacity=0] (7.14,-3.43) -- (0,0) -- (7.14,3.43) -- cycle    ;
%Shape: Ellipse [id:dp5733943324242119]
\draw   (194.69,42.05) .. controls (194.69,37.86) and (198.09,34.46) .. (202.28,34.46) .. controls (206.47,34.46) and (209.87,37.86) .. (209.87,42.05) .. controls (209.87,46.24) and (206.47,49.64) .. (202.28,49.64) .. controls (198.09,49.64) and (194.69,46.24) .. (194.69,42.05) -- cycle ;
%Shape: Ellipse [id:dp4982173340393281]
\draw   (198.69,97.05) .. controls (198.69,92.86) and (202.09,89.46) .. (206.28,89.46) .. controls (210.47,89.46) and (213.87,92.86) .. (213.87,97.05) .. controls (213.87,101.24) and (210.47,104.64) .. (206.28,104.64) .. controls (202.09,104.64) and (198.69,101.24) .. (198.69,97.05) -- cycle ;
%Shape: Ellipse [id:dp6443249330347911]
\draw   (145.69,58.05) .. controls (145.69,53.86) and (149.09,50.46) .. (153.28,50.46) .. controls (157.47,50.46) and (160.87,53.86) .. (160.87,58.05) .. controls (160.87,62.24) and (157.47,65.64) .. (153.28,65.64) .. controls (149.09,65.64) and (145.69,62.24) .. (145.69,58.05) -- cycle ;
%Shape: Ellipse [id:dp2962874847585506]
\draw   (151.69,79.05) .. controls (151.69,74.86) and (155.09,71.46) .. (159.28,71.46) .. controls (163.47,71.46) and (166.87,74.86) .. (166.87,79.05) .. controls (166.87,83.24) and (163.47,86.64) .. (159.28,86.64) .. controls (155.09,86.64) and (151.69,83.24) .. (151.69,79.05) -- cycle ;
%Straight Lines [id:da08505254887451863]
\draw    (198.69,94.05) -- (168.68,83.08) ;
\draw [shift={(165.87,82.05)}, rotate = 20.08] [fill={rgb, 255:red, 0; green, 0; blue, 0 }  ][line width=0.08]  [draw opacity=0] (7.14,-3.43) -- (0,0) -- (7.14,3.43) -- cycle    ;
%Straight Lines [id:da8488856387468098]
\draw    (105.59,78.59) -- (148.69,79.02) ;
\draw [shift={(151.69,79.05)}, rotate = 180.57] [fill={rgb, 255:red, 0; green, 0; blue, 0 }  ][line width=0.08]  [draw opacity=0] (7.14,-3.43) -- (0,0) -- (7.14,3.43) -- cycle    ;
%Straight Lines [id:da9679144897377887]
\draw    (145.69,58.05) -- (105.98,71.6) ;
\draw [shift={(103.14,72.57)}, rotate = 341.16] [fill={rgb, 255:red, 0; green, 0; blue, 0 }  ][line width=0.08]  [draw opacity=0] (7.14,-3.43) -- (0,0) -- (7.14,3.43) -- cycle    ;
%Straight Lines [id:da895747160462798]
\draw    (194.69,42.05) -- (114.86,47.83) ;
\draw [shift={(111.87,48.05)}, rotate = 355.86] [fill={rgb, 255:red, 0; green, 0; blue, 0 }  ][line width=0.08]  [draw opacity=0] (7.14,-3.43) -- (0,0) -- (7.14,3.43) -- cycle    ;
%Straight Lines [id:da9678052539321507]
\draw    (195.69,45.05) -- (163.73,55.15) ;
\draw [shift={(160.87,56.05)}, rotate = 342.47] [fill={rgb, 255:red, 0; green, 0; blue, 0 }  ][line width=0.08]  [draw opacity=0] (7.14,-3.43) -- (0,0) -- (7.14,3.43) -- cycle    ;
%Straight Lines [id:da12597861271491162]
\draw    (202.28,49.64) -- (205.98,86.48) ;
\draw [shift={(206.28,89.46)}, rotate = 264.26] [fill={rgb, 255:red, 0; green, 0; blue, 0 }  ][line width=0.08]  [draw opacity=0] (7.14,-3.43) -- (0,0) -- (7.14,3.43) -- cycle    ;
%Straight Lines [id:da05573449739087133]
\draw    (197.14,47.57) -- (166.5,71.71) ;
\draw [shift={(164.14,73.57)}, rotate = 321.77] [fill={rgb, 255:red, 0; green, 0; blue, 0 }  ][line width=0.08]  [draw opacity=0] (7.14,-3.43) -- (0,0) -- (7.14,3.43) -- cycle    ;
%Shape: Ellipse [id:dp18633067687908644]
\draw  [color=blue  ,draw opacity=1 ][line width=1]  (2.14,95.57) .. controls (2.14,43.66) and (58.55,1.57) .. (128.14,1.57) .. controls (197.73,1.57) and (254.14,43.66) .. (254.14,95.57) .. controls (254.14,147.49) and (197.73,189.57) .. (128.14,189.57) .. controls (58.55,189.57) and (2.14,147.49) .. (2.14,95.57) -- cycle ;
%Curve Lines [id:da9916896653167908]
\draw [color=blue  ,draw opacity=1 ][line width=1]    (30.14,154.57) .. controls (79.86,141.57) and (142.86,99.57) .. (133.86,2) ;
%Curve Lines [id:da16207253703394464]
\draw [color=blue  ,draw opacity=1 ][line width=1]    (233.86,146.57) .. controls (224.86,122.57) and (151.86,96.57) .. (115,94.57) ;

% Text Node
\draw (19,16) node [anchor=north west][inner sep=0.75pt]    {$\textcolor{blue}{\mathbf{A}}$};
% Text Node
\draw (228,16) node [anchor=north west][inner sep=0.75pt]    {$\textcolor{blue}{\mathbf{B}}$};
% Text Node
\draw (217,168.4) node [anchor=north west][inner sep=0.75pt]    {$\textcolor{blue}{\mathbf{C}}$};


% Define nodes in the condensed graph
\node[graph node, fill=blue!30, draw=black] (A) at (340,100) {A};
\node[graph node, fill=blue!30, draw=black] (B) at (400,50) {B};
\node[graph node, fill=blue!30, draw=black] (C) at (400,150) {C};

% Draw edges in the quotient graph
\draw[->, thick] (A) to[out=0, in=90] (B);
\draw[->, thick] (B) to[out=180, in=270] (A);
\draw[->, thick] (C) -- (A);

\end{tikzpicture}
    \caption{Esempio di grafo quoziente di un grafo orientato}
    \label{fig:quotient-graph-example}
\end{figure}

Le relazioni di equivalenza, cos\'{\i} come le partizioni, possono essere comparate tra
loro secondo il concetto di raffinamento: una relazione di equivalenza $R_1$ si dice \textbf{pi\`u fine}
(in inglese \textbf{finer}) di un'altra relazione di equivalenza $R_2$ se ogni classe di equivalenza di $R_1$ \`e
contenuta in una classe di equivalenza di $R_2$.
In tal caso si dice che $R_2$ \`e \textbf{pi\`u grezza} (in inglese \textbf{coarser}) di $R_1$, in quanto ogni classe
di equivalenza di $R_2$ pu\`o essere ottenuta come l'unione di classi di equivalenza di $R_1$. \newline

Per questo \`e interessante notare che tale concetto di finezza pu\`o essere facilmente esteso ai grafi quoziente:
\begin{itemize}
    \item Ogni grafo pu\`o banalmente considerarsi come il grafo quoziente di se stesso rispetto
    alla relazione di equivalenza di ugualianza, in quanto ogni nodo di un grafo \`e uguale unicamente a se stesso.
    La relazione di equivalenza di ugualianza, infatti, \`e la relazione di equivalenza pi\`u fine, e per questo
    genera il grafo quoziente pi\`u fine possibile a partire da qualunque grafo.
    \item Analogamente, il grafo composto di un unico nodo e nessun arco risulta il grafo quoziente di ogni grafo
    rispetto alla relazione di equivalenza universale, che mette in relazione qualsiasi coppia di elementi ed
    identifica tutti i nodi di qualunque grafo in un unico blocco.
    La relazione di equivalenza universale, infatti, \`e la relazione di equivalenza pi\`u grezza e, come \`e
    intuibile pensare, genera il grafo quoziente pi\`u grezzo possibile a partire da qualunque grafo.
\end{itemize}

Una particolare relazione di equivalenza che ben si presta alla definizione di un grafo quoziente \`e la relazione di
mutua raggiungibilit\`a tra nodi di un grafo, che ne definisce le componenti fortemente connesse.
Il grafo quoziente di un grafo rispetto a tale relazione di equivalenza prende il nome di
\textbf{condensazione} (o grafo delle componenti fortemente connesse), e si dimostra essere un grafo diretto aciclico.

\begin{figure}[h]
    \centering
    \begin{tikzpicture}[>={Stealth}]

% Define nodes in the original graph
\node[] (1) at (0,0) {$v_1$};
\node[] (2) at (0.5,1) {$v_2$};
\node[] (3) at (1,0) {$v_3$};

\node[] (4) at (1,-2) {$v_4$};

\node[] (5) at (3,-1) {$v_5$};
\node[] (6) at (4,-1.5) {$v_6$};

\node[] (7) at (3.5,1.8) {$v_7$};
\node[] (8) at (3.25,0.8) {$v_8$};
\node[] (9) at (4.5,0.8) {$v_9$};
\node[] (10) at (4.75,1.8) {$v_{10}$};

% Draw edges in the original graph
\draw[->] (1) -- (2);
\draw[->] (2) -- (3);
\draw[->] (2) -- (7);
\draw[->] (3) -- (1);
\draw[->] (3) -- (5);
\draw[->] (3) -- (7);
\draw[->] (4) -- (1);
\draw[->] (4) -- (3);
\draw[->] (5) -- (6);
\draw[->] (6) to[out=225, in=270] (5);
\draw[->] (5) -- (8);
\draw[->] (6) -- (9);
\draw[->] (7) -- (8);
\draw[->] (7) -- (10);
\draw[->] (8) -- (9);
\draw[->] (9) -- (7);
\draw[->] (10) -- (9);

% Draw subsets
\begin{scope}[on background layer]
    \node[ellipse, draw=yellow, line width=2pt, fit=(1) (2) (3)] {};
    \node[ellipse, draw=yellow, line width=2pt, fit=(4)] {};
    \node[ellipse, draw=yellow, line width=2pt, fit=(5) (6)] {};
    \node[ellipse, draw=yellow, line width=2pt, fit=(7) (8) (9) (10)] {};
\end{scope}

% Draw condensed graph on the left
\node[fill=yellow!30, draw=black] (A) at (7.5,0.5) {A};
\node[fill=yellow!30, draw=black] (B) at (8,-1) {B};
\node[fill=yellow!30, draw=black] (C) at (9,-0.25) {C};
\node[fill=yellow!30, draw=black] (D) at (9.25,1.25) {D};

\draw[->, thick] (B) -- (A);
\draw[->, thick] (A) -- (C);
\draw[->, thick] (A) -- (D);
\draw[->, thick] (C) -- (D);

\end{tikzpicture}
    \caption{Esempio di condensazione di un grafo orientato}
    \label{fig:condensation-example}
\end{figure}

In figura~\ref{fig:condensation-example} \`e mostrato un esempio di grafo orientato sulla sinistra, in cui le componenti
fortemente connesse sono evidenziate in giallo, e al sua condensazione sulla destra. \newline
Si noti che il grafo condensato, in quanto aciclico, non contiene cicli semplici. \newline
Se si volesse aggiungere un arco affinch\`e il grafo condensato contenesse un ciclo, ad esempio aggiungendo un arco
uscente da un nodo nella compoente $C$ e entrante in un nodo nella componente $A$, si otterrebbe una nuova componente
fortemente connessa data dall'unione delle componenti $A$, $B$ e $C$, ovvero le componenti i cui corrispondenti nodi
nel grafo condensato sarebbero contentenuti in un ciclo semplice.