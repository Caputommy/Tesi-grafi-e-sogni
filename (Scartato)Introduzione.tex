\chapter*{Introduzione}\addcontentsline{toc}{chapter}{Introduzione}

Numerosi e ragguardevoli sono stati i traguardi raggiunti dagli strumenti di elaborazione automatica di testi
scritti sviluppati ed affinati negli ultimi decenni.
In particolare, la branca dell'intelligenza artificiale dell'elaborazione automatica del linguaggio naturale
(\textit{Natural Language Processing} o \textit{NLP}) ha
visto una crescita esponenziale negli ultimi anni, grazie all'impiego di tecniche di deep
learning e all'incremento della potenza di calcolo a disposizione.
Tuttavia, nonostante i progressi compiuti circa le capacit\`a, la comprensione del testo scritto rimane un compito
complesso per i sistemi automatici, che richiedono enormi quantitativi di dati annotati per poter apprendere
modelli di linguaggio sufficientemente accurati.
Si pensi che i dataset per l'addestramento di modelli di linguaggio come \textit{GPT-3} si sono rapidamente espansi,
culminando in dimensioni dell'ordine del trilione di parole~\cite{DBLP:journals/corr/abs-2005-14165}. \newline


D'altra parte, l'applicazione di queste tecniche di analisi automatica di testi scritti o parlati sui sogni hanno
gi\`a trovato applicazioni in ambito psicologico e psichiatrico, applicando tecniche di NLP su piccoli corpus di testi
di sogni ~\cite{ALTSZYLER2017178}, mentre la rappresentazione di testi sui sogni attraverso grafi si \`e rivelato
un'utile strumento a supperto della diagnosi e predizione di disturbi come la schizofrenia o il disturbo
bipolare ~\cite{mota2014graph,mota2017thought}~. \newline

In questa tesi si discuter\`a di come una struttura di grafi a pi\`u livelli possa essere utilizzata
per rappresentare e analizzare piccoli dataset di testi provenienti da trascrizioni di sogni e come essa possa essere
sfruttata per realizzare dei modelli di linguaggio minimali rappresentativi di un particolare sognatore, dimostrando
come informazioni legate alla semantica delle parole possano essere estrapolate a partire da aspetti sintattici.
\newline

\section*{Motivazione}
Sebbene gli strumenti di NLP siano stati ampiamente utilizzati per l'analisi sintattica e semantica di testi scritti,
le motivazioni che spingono alla realizzazione di una struttura dati applicabile alla trasposizione di sogni in
semplici modelli di linguaggio sono legate al tentativo di individuare i rapporti sintattici e semantici tra parole e
contesti di parole in relazione alla specifica persona, ovvero allo spazio semantico di un sognatore. \newline

%Attenzione: usato dopo
L'estrapolazione di informazioni legate al significato delle parole a partire da aspetti sintattici \`e un
principio fondamentale della semantica computazionale, noto come \textit{ipotesi distribuzionale}, che si basa sul
principio per cui il significato di una parola \`e determinato dal suo contesto di utilizzo, e che le parole che
appaiono nello stesso contesto tendono ad avere significati simili.\newline

Lo spazio semantico di un sognatore pu\`o essere rappresentato, quindi, attraverso una struttura basata su grafi in cui
i nodi rappresentino le parole o i contesti di parole presenti nei sogni e gli archi siano ricavati dalle relazioni
sintattiche tra di esse, come l'immediata vicinanza o la co-occorrenza. \newline

 realizzare una struttura dati generica per la rappresentazione di una
gerarchia di grafi a pi\`u livelli, dove ogni livello rappresenti una diversa astrazione del grafo iniziale, e i
grafi ai livelli inferiori possano essere ottenuti dall'espansione ricorsiva dei nodi ai livelli superiori. \newline

Da questa necessit\`a nasce l'idea di definire e realizzare nella maniera pi`u generale possibile una struttura
gerarchica di grafi a pi`u livelli dove ogni livello rappresenti una diversa astrazione del grafo inziale,
e i grafi ai livelli inferiori possono essere ottenuti dall’espansione ricorsiva dei nodi ai livelli superiori,
permettendo espansioni locali o globali.
Ci`o pu`o essere utile per descrivere caratteristiche strutturali di grafi grandi e sparsi a partire
dall’individuazione di schemi topologici che siano significativi, che essi siano, ad esempio, cricche, componenti
fortemente connesse, circuiti, stelle o cammini, e fornire operazioni che permettano di operare ad un determinato
livello di astrazione nella gerarchia, come ottenere il grafo corrispondente ad un nodo, il calcolo della distanza
o la modifica della struttura.

\section*{Obiettivi}
L'obiettivo principale di questa tesi \`e, quindi, proporre una definizione formale della struttura dati astratta del
\textit{Grafo Muli-Livello} e delle operazioni che possono essere eseguite su di essa, nonch\'e di valutarne
complessit\`a computazionale e spaziale. \newline

Il secondo obiettivo \`e quello di analizzarne le possibili applicazioni, in particolare per la
rappresentazione di spazi di parole e contesti in relazione ai sogni trascritti da un determinato sognatore. \newline

\section*{Struttura della Tesi}

La tesi \`e strutturata come segue:
\begin{itemize}
    \item Nel Capitolo~\ref{chap:background} verranno presentati i concetti di base relativi alla teoria dei grafi,
    con particolare attenzione agli aspetti legati al partizionamento e agli approcci multilivello esistenti per la
    risoluzione di problemi di partizionamento su grafi.
    \item Nel Capitolo~\ref{chap:multilevel} verranno illustrati gli algoritmi per l'individuazione di pattern
    strutturali in grafi utili per la costruzione di una gerarchia di grafi a pi\`u livelli.
    \item Nei Capitoli~\ref{chap:multilevel} verr\`a presentata e definita la struttura dati del Grafo Multi-Livello,
    le operazioni che possono essere eseguite su di essa e i relativi algoritmi
    \item Nei Capitoli~\ref{chap:applications} verranno discusse le possibili applicazioni del Grafo Multi-Livello,
    evidenziandone limiti e punti di forza.
    Particolare attenzione sar\`a rivolta all'ambito della trascrizione dei sogni, e verranno discussi i risultati
    preliminari ottenuti applicando la struttura dati a dataset di sogni di esempio.
\end{itemize}